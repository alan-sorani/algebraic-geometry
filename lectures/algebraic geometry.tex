\documentclass[10pt,a4paper,twoside,openany,hidelinks]{book}

\usepackage{maths}
\usepackage{stylish}

\usepackage{tkz-graph}
\GraphInit[vstyle = Shade]

\title{Lecture Notes to Fundamental Concepts in Algebraic Geometry \\ \large{Spring 2020, Hebrew University of Jerusalem}}
\author{Ariel Shnidman\\ \large{Typed by Elad Tzorani}}
\date{\today}

\usepackage{lipsum}
\usepackage{stmaryrd}

\tikzset{EdgeStyle/.style   = {thick,
                               double          = orange,
                               double distance = 1pt}}

\tikzset{
    labl0/.style={anchor=south, rotate=-30, inner sep=.5mm}
}

\tikzset{
    labl1/.style={anchor=south, rotate=30, inner sep=.5mm}
}

\tikzset{
    labl2/.style={anchor=south, rotate=60, inner sep=.5mm}
}

\tikzset{
    labl3/.style={anchor=south, rotate=90, inner sep=.5mm}
}

\tikzset{
    labl4/.style={anchor=south, rotate=-90, inner sep=.5mm}
}

\usepackage{scalerel,stackengine}
\stackMath
\newcommand\widecheck[1]{%
\savestack{\tmpbox}{\stretchto{%
  \scaleto{%
    \scalerel*[\widthof{\ensuremath{#1}}]{\kern-.6pt\bigwedge\kern-.6pt}%
    {\rule[-\textheight/2]{1ex}{\textheight}}%WIDTH-LIMITED BIG WEDGE
  }{\textheight}% 
}{0.5ex}}%
\stackon[1pt]{#1}{\scalebox{-1}{\tmpbox}}%
}
\parskip 1ex

\newcommand{\from}{\leftarrow}

\usepackage[nottoc]{tocbibind}

\begin{document}
\frontmatter
\frontpage{front}{0.8\textwidth}{}
\tableofcontents
\countlectures
\newpage

\chapter*{Preface}
\addcontentsline{toc}{chapter}{Preface} \markboth{Preface}{}

\section*{Technicalities}
\addcontentsline{toc}{section}{Technicalities} %\markboth{Technicalities}{}

These aren't formal notes related to the course and henceforward there is \emph{absolutely no guarantee} that the recorded material is in correspondence with the course expectations, or that these notes lack any mistakes.\\
In fact, there probably are mistakes in the notes! I would highly appreciate if any comments or corrections were sent to me via email at \href{mailto:tzorani.elad@gmail.com}{tzorani.elad@gmail.com}.\\
Elad Tzorani.

\mainmatter

\chapter*{Goals of Algebraic Geometry}

The main goals of algebraic geometry are the following.

\begin{itemize}
\item To classify algebraic varieties up to birational isomorphism.
Either by enumeration of the varieties or classification by properties.
\item
Be able to tell whether two algebraic varieties are (birationally) isomorphic. This is done through different tools:
\begin{itemize}
\item Cohomology (related to the Hodge Conjecture).
\item Vector bundles (related to K-theory).
\item Algebraic Cycles (subvarieties).
\item Coherent sheaves (derived category).
\item Topology.
\end{itemize}
\end{itemize}

\chapter{Sheaves}

Let $X \in \Top$.

\begin{definition}
Let $\Top_X$ be the category of open sets on $X$ with inclusions as morphisms.
\end{definition}

\begin{notation}
Denote by $\Ab$ the category of abelian groups.
\end{notation}

\begin{definition}[Presheaf]
A \stress{presheaf (of groups)} on $X$ is a contravariant functor
\[F \colon \Top_X \to \Ab \text{.}\]
\end{definition}

Concretely, to give a presheaf we must give the following:
\begin{enumerate}
\item For each $U \subseteq X$ open, an abelian group $F\prs{U}$.
\item For inclusion $V \rmono U$ of open sets, a group homomorphism $F_{U,V} \colon F\prs{U} \to F\prs{V}$.
\end{enumerate}
Under the conditions:
\begin{enumerate}
\item $F_{U,U} = \id_{F\prs{U}}$.
\item If $W \rmono V \rmono U$ then $W \rmono U$ so the induced maps
\[
\begin{tikzcd}
F\prs{U} \arrow[r] \arrow[rr, bend right = 30] & F\prs{V} \arrow[r] & F\prs{W}
\end{tikzcd}
\]
commute.
\end{enumerate}

\begin{examples}
\begin{enumerate}
\item Let \[F\prs{U} \ceq \hom\prs{U, \RR} = \set{f \colon U \to \RR} \text{.}\]
If $V \subseteq U$ are open,
\begin{align*}
F\prs{U} &\xrightarrow{F_{U,V}} F\prs{V} \\
f &\mapsto \rest{f}{V} \hphantom{F\prs{V}} \text{.}
\end{align*}

This is actually also a presheaf of rings on $X$.

\item \stress{The constant presheaf:} Let $A \in \Ab$ and define
\[F\prs{U} \ceq A, \quad F_{U,V} = \id_A \text{.}\]

\item Let $k$ be an algebraically closed field and $X$ an algebraic variety over $k$. We can think $X \subseteq \PP_k^n$.

Define $\Oo_X\prs{U} \ceq \set{f \colon U \to k}{\text{$f$ is regular}}$.
By ``regular'' we mean that for all $P \in U$ there's a neighbourhood $V \subseteq U$ of $P$ such that
\[\rest{f}{V} = \frac{g}{h}\]
with
$g,h \in k\brs{x_0, \ldots, x_n}$.

\item Let $X = \PP^1$.
Then
$\Oo_{\PP^1}\prs{\PP^1} = k$.
Also
$\Oo_{\PP^1} \prs{\PP^1 \setminus \set{0}} = \set{\frac{g\prs{x,y}}{h\prs{x,y}}}{\substack{g,h \in k\brs{x,y}\\\forall \prs{x,y} \neq \prs{0,1} \colon h\prs{x,y} \neq 0}} = \set{\frac{g\prs{x,y}}{x^{\deg\prs{g}}}}$.
\end{enumerate}
\end{examples}

\begin{definition}[Section, Restriction]
For any $F$ as above, $s \in F\prs{U}$ is called a \stress{section}.
$F_{U,V}\prs{s}$ is written as $\rest{s}{V}$ and called \stress{restriction of $s$ to $V$}.
\end{definition}

\begin{definition}[Global Section]
$F\prs{X} \ceq \Gamma\prs{X,F}$ is the \stress{global section of $F$}.
\end{definition}

\begin{remark}
Why ``section''? Think about $\Oo_X\prs{U}$ and view the diagram
\[
\begin{tikzcd}
U \times k \arrow[d,"\pi_U"] \arrow[r, hook] & X \times k \arrow[d,"\pi"] \\
U \arrow[u, "\tilde{s}", bend left = 40] \arrow[r, hook] & X
\end{tikzcd}
\]
and call $X \times k$ the \stress{trivial line bundle}.

Take $s \in \Oo_X\prs{U}$ which induces a map
\begin{align*}
\tilde{s} \colon U &\to U \times k \\
x &\mapsto \prs{X, s\prs{x}} \text{.}
\end{align*}
Then $\tilde{s}$ is a {section of $\pi$ over $U$}
in the sense that $\pi_U \circ \tilde{s} = \id_U$.
\end{remark}

The idea of sheaves is that global sections are difficult to study topologically, but $F\prs{U}$ is easier to study for $U$ small, and computation on smaller $U$ can imply properties of $F\prs{U}$.

\begin{definition}[Sheaf]
A \stress{sheaf (over an abelian group)} $F$ is a presheaf over an abelian group such that the following holds.
\begin{enumerate}
\item If $U \subseteq X$ is open and $\set{V_i}_{i \in I}$ is an open cover of $U$, and if $s \in F\prs{U}$ such that $\forall i \in I \colon \rest{s}{V_i} = 0$ then $s = 0$.
\item If $U \subseteq X$ is open and $\set{V_i}_{i \in I}$ is an open cover, and if \[\forall i \in I \exists s_i \in F\prs{V_i} \forall i,j \in I \colon \rest{s_i}{V_i \cap V_j} = \rest{s_j}{V_i \cap V_j}\]
then
\[\exists ! s \in F\prs{U} \forall i \in I \colon \rest{s}{V_i} = s_i \text{.}\]
\end{enumerate}
\end{definition}

\begin{remark}
\begin{enumerate}
\item
We could equivalently remove the uniqueness requirement in the second condition as it's implied by the first condition.
\item
With the uniqueness condition, the first condition is implied by the second.
\end{enumerate}
\end{remark}

\begin{remark}
There is a categorical definition of a sheaf over any (not necessarily abelian) category, which might be defined later in these notes.
\end{remark}

\begin{examples}
\begin{enumerate}
\item $\Oo_x$ is a sheaf on $X$.
\item $F\prs{U} = A \in \Ab$ is \stress{not} in general a sheaf unless $A = 0$.

If $X$ has two connected components one can choose different elements of $A$ on two disjoint open sets, which agree on the (empty) intersection, but which we can't lift to the union.
\item Give $A \in \Ab$ the discrete topology.
Define $F_A\prs{U} \ceq \hom_{\mrm{cont}}\prs{U,A}$ the set of continuous functions $U \to A$.
This is a sheaf.

This is sometimes called the \stress{constant sheaf}, although it isn't actually constant.
If $U \subseteq A$ is connected then $F_A\prs{U} \cong A$.
\end{enumerate}
\end{examples}

\begin{definition}[Stalk of a Sheaf]
Let $p \in X$ and $F$ a presheaf on $X$.
The \stress{stalk of $F$ at $p$} is
\[F_p \ceq \lim_{\substack{\longrightarrow \\ p \in U \subseteq X}} F\prs{U} = \quot{\coprod_{p \in U \subseteq X} F\prs{U}}{\substack{\prs{U, s \in F\prs{U}} \sim \prs{V, t \in F\prs{V}} \\ \text{if $\rest{s}{U \cap V} = \rest{t}{U \cap V}$} \hphantom{lala}\text{.}}} \]
\end{definition}

\begin{definition}[Morphism of Presheaves]
Let $F,G$ be presheaves on $X$, a morphism $f \colon F \to G$ is a natural transformation.
\end{definition}

\begin{remark}
Concretely, a morphism $f \colon F \to G$ of presheaves is a group homomorphism
\[f_U \colon F\prs{U} \to G\prs{U}\]
for every $U \subseteq X$ open such that the diagram
\[
\begin{tikzcd}
F\prs{U} \arrow[r, "f_U"] \arrow[d, "\mrm{res}", swap] & G\prs{U}\arrow[d, "\mrm{res}"] \\
F\prs{V} \arrow[r, "f_V", swap] & G\prs{V}
\end{tikzcd}
\]
commutes for every opens $V \subseteq U$.
\end{remark}

\begin{definition}[Isomorphism of Presheaves]
A morphism of presheaves is an isomorphism if it has a two-sided inverse.
\end{definition}

\begin{notation}
If $\phi \colon F \to G$ is a morphism there's a an induced map on stalks
\begin{align*}
\phi_p \colon F_p \to G_p
\end{align*}
for every $p \in X$ where we remind
\begin{align*}
F_p &= \lim_{\substack{\longrightarrow \\ p \in U}} F\prs{U} \\
G_p &= \lim_{\substack{\longrightarrow \\ p \in U}} G\prs{U} \text{.}
\end{align*}
\end{notation}

\begin{definition}[Morphism of Sheaves]
A morphism of sheaves is a morphism between the respective presheaves. 
\end{definition}

\begin{remark}
The above implies that presheaves over $X$ are a full subcategory of sheaves over $X$.
\end{remark}

\begin{proposition}
Let $\phi \colon F \to G$ be a morphism of sheaves.
Then $\phi$ is an isomorphism iff $\phi_p$ is an isomorphism for every $p \in X$.
\end{proposition}

\begin{remark}
The above does \stress{not} say that $F \cong G$ iff $F_p \cong G_p$ for every $p \in X$.

This isn't true.
\end{remark}

\begin{proof}
\begin{description}
\item[$\Rightarrow$:] This is straightforward.
\item[$\Leftarrow$:]
\begin{description}
\item[Injectivity:]
Let $U \subseteq X$ open and $s \in F\prs{U}$ such that $\phi\prs{s} = 0$. WTS (want to show) $s = 0$.

Observe the following commutative diagram.
\[
\begin{tikzcd}
F\prs{U} \arrow[r, "\phi"] \arrow[d] & G\prs{U} \arrow[d] \\
F_p \arrow[r, swap, "\phi_p"] & G_p
\end{tikzcd}
\]
$\phi_p\prs{s_p} = 0$ implies by injectivity $s_p = 0$. Then
\[\exists p \in W_p \subseteq U \colon \rest{s}{W_p} = 0 \text{.}\]
We can cover $U$ by these $\set{W_p}_{o \in U}$ is a cover over $U$ and by the first sheaf condition we get $s = 0$.
\item[Surjectivity:]
%26.3.2020
We want to show that $\phi_U \colon F\prs{U} \to G\prs{U}$ is surjective for every $U \subseteq X$ open.

Let $s \in G\prs{U}$, we consider the following commutative diagram for every $p \in X$.
\[
\begin{tikzcd}
F\prs{U} \arrow[r,"\phi"] \arrow[d] & G\prs{U} \arrow[d] \\
F_p \arrow[r, "\phi_p"] & G_p
\end{tikzcd}
\]
Take $s \in G\prs{U}$ from which we get $s_p \in G_p$.
Then by surjectivity of $\phi_p$ there's $\tilde{t}_p \in F_p$ such that $\phi_p\prs{\tilde{t}_p} = s_p$.
Then there's $p \in W_p \subseteq U$ and $t_p \in F\prs{W_p}$ such that
$\phi_{W_p}\prs{t_p} = \rest{s}{W_p}$.
We want to glue the $t_p$ to get $t \in F\prs{U}$.

Check that the overlaps agree by showing that for $P,Q \in U$, $t_P, t_Q$ agree.
Indeed,
\begin{align*}
\phi\prs{\rest{t_P}{W_P \cap W_Q}} &= \rest{\phi\prs{t_P}}{W_P \cap W_Q}
\\&=
\rest{\prs{\rest{s}{W_P}}}{W_P \cap W_Q}
\\&= \brs{\text{same computation}}
\\&= \phi\prs{\rest{t_Q}{W_P \cap W_Q}}
\end{align*}
and by injectivity we get
\[\forall P,Q \in U \colon \rest{t_P}{W_P \cap W_Q} = \rest{t_Q}{W_P \cap W_Q} \text{.}\]

Since $F$ is a sheaf, there's $t \in F\prs{U}$ such that $\rest{t}{W_p} = t_p$.

We want to check that $\phi\prs{t} = s \in G\prs{U}$. Because $G$ is a sheaf we can check this locally.
\begin{align*}
\rest{\phi\prs{t}}{W_p} &= \phi\prs{\rest{t}{W_p}}
\\&=
\phi\prs{t_p}
\\&=
\rest{s}{W_p}
\end{align*}
Then by uniqueness of the restrictions $\phi\prs{t} = s$, so $\phi$ is indeed surjective.
\end{description}
\end{description}
\end{proof}

\begin{definition}
Let $\phi \colon F \to G$ a morphism of presheaves of abelian groups. Define the following presheaves.
\begin{align*}
\prs{\ker \phi}{U} &\ceq \ker\prs{\phi_U} \subseteq F\prs{U} \\
\prs{\widetilde{\coker}\phi}\prs{U} &\ceq \coker\prs{\phi_U} = \quot{G\prs{U}}{\phi\prs{F\prs{U}}} \\
\prs{\widetilde{\im}\phi}\prs{U} &= \im\prs{\phi_U} \subseteq G\prs{U}
\end{align*}
$\ker \phi$ is a sheaf if $F,G$ are. The other two aren't necessarily sheaves in this case.
\end{definition}

In order to define cokernel and image which are sheaves, we produce a functor from presheaves to sheaves, which is adjoint to the forgetful functor.

\begin{proposition}[Sheafification]
Let $F$ be a presheaf on $X$, there's a sheaf $F^+$ and a morphism $\theta \colon F \to F^+$ such that the following holds.

If $G$ is a sheaf and $F \to G$ is a morphism of presheaves, there's a unique morphism $\psi \colon F^+ \to G$ such that the following diagram commutes.
\[
\begin{tikzcd}
F \arrow[rr] \arrow[dr, "\theta", swap] & & G \\
& F^+ \arrow[ru, "\psi", swap, dotted] &
\end{tikzcd}
\]

Moreover, $\prs{F^+, \theta}$ is unique up to unique isomorphism. I.e if $F \xrightarrow{\theta'} F'^+$ is another such object, there's a unique isomorphism such that the following commutes.
\[
\begin{tikzcd}
F \arrow[r, "\theta'"] \arrow[dr, "\theta"] & \prs{F'}^+ \arrow[d, "\sim" labl4, dotted] \\
& F^+
\end{tikzcd}
\]
\end{proposition}

\begin{proof}
Define $F^+$ as follows. For $U \subseteq X$ open let
\[F^+\prs{U} = \set{s \colon U \to \coprod_{p \in U} F_p}{\substack{\forall p \in U \colon s\prs{p} \in F_p \\ \forall p \in U \exists p \in V \subseteq U \exists t \in F\prs{V} \forall Q \in V \colon t_Q = s\prs{Q} \in F_Q}} \text{.}\]
Check that this is a sheaf.

Define
\begin{align*}
\theta \colon F &\to F^+ \\
F\prs{U} &\to F^+\prs{U} \\
s \in F\prs{U} &\mapsto \brs{\substack{U \to \coprod F_p \\
p \mapsto s_p}} \text{.}
\end{align*}
Check that $\prs{F^+, \theta}$ satisfies the universal property.
\end{proof}

We have the following properties.

\begin{proposition}
\begin{enumerate}
\item If $F$ is a sheaf, $\theta \colon F \to F^+$ is an isomorphism.

This follows from the universal property.
\item If $p \in X$ then $F_p \cong \prs{F^+}_p$ via $\theta$.
\end{enumerate}
\end{proposition}

\begin{definition}[Subsheaf]
Let $F$ a sheaf, a \stress{subsheaf} $F'$ is a sheaf such that the diagram
\[
\begin{tikzcd}
F'\prs{U} \arrow[r, hook] \arrow[d] & F\prs{U} \arrow[d] \\
F'\prs{V} \arrow[r,hook] & F\prs{V}
\end{tikzcd}
\]
commutes for all $V \subseteq U$.
\end{definition}

\begin{definition}[$\coker, \im$]
Let $\phi \colon F \to G$ a morphism of sheaves then $\ker \phi$ is already a sheaf.

Define
\begin{align*}
\im \phi &\ceq \prs{\widetilde{\im} \phi}^+ \\
\coker\phi &\ceq \prs{\widetilde{\coker} \phi}^+ \text{.}
\end{align*}
\end{definition}

\begin{remark}
By the universal property $\im \phi \rmono G$ is an injection.
\end{remark}

\begin{definition}[Injective Morphism]
We say $\phi$ is \stress{injective} if $\ker \phi = 0$, i.e. if $\phi_U$ is injective for every $U$.
\end{definition}

\begin{definition}[Surjective Morphism]
We say $\phi$ is \stress{surjective} if $\im \phi = G$.
\end{definition}

\begin{remark}
Surjectivity is \stress{not} equivalent to $\phi_U$ being surjective for all $U$.
\end{remark}

\begin{definition}[Exact Sequence]
Say a sequence of sheaves
\[\cdots \rightarrow F_{-2} \xrightarrow{\phi_{-2}} F_{-1} \xrightarrow{\phi_{-1}} F_0 \xrightarrow{\phi_0} F_1 \xrightarrow{\phi_1} \cdots\]
is \stress{exact} if $\ker \phi_i = \im \phi_{i-1}$.

If $F' \subseteq F$, define $\quot{F'}{F}$ to be the sheafification of the presheaf
\[U \mapsto \quot{F'\prs{U}}{F\prs{U}} \text{.}\]
\end{definition}

\begin{definition}[Pushforward and Inverse Image Sheaves]
Let $f \colon X \to Y$ be a continuous map of topological spaces.

Let $F$ be a sheaf on $X$, we define for $V \subseteq Y$ open
\[\prs{f_* F}\prs{V} \ceq F\prs{f^{-1}\prs{V}} \text{.}\]
Then $f_* F$ is a sheaf on $Y$
and is called the \stress{pushforward of $F$}.

If $f \colon X \to Y$ and $G$ a sheaf on $Y$. We define for $U \subseteq X$ open
\[\prs{f^{-1}G}\prs{U} = \prs{\lim_{\substack{\longrightarrow \\ V \supseteq f\prs{U}}} G\prs{V}}^+ \text{.}\]
Then $f^{-1}G$ is a sheaf and called the \stress{inverse image sheaf}.
\end{definition}

\begin{remark}
$f_*$ and $f^{-1}$ are functors between sheaves on $X$ and $Y$.
\end{remark}

\begin{exercise}
$f_*$ and $f^{-1}$ are adjoint functors.
I.e.
\[\hom_X\prs{f^{-1}G, F} \cong \hom_Y\prs{G, f_* F}\]
naturally.
\end{exercise}

\begin{notation}
Let $Z \subseteq X$ and $i \colon Z \rmono X$ the inclusion map. For a sheaf $F$ on $X$ we write $\rest{F}{Z}$ instead of $i^{-1} F$.
\end{notation}

\begin{exercise}
Assume $F\prs{A} = A \in \Ab$ and that $\theta U \subseteq X$.
Let
\[F_A\prs{U} \ceq \hom_{\mrm{cont}}\prs{U,A} \text{,}\]
we get
$F_A \cong F^+$.
\end{exercise}

\chapter{Schemes}

\section{Motivation} 

We notice that in in classical algebraic geometry the interesting object of an algebraic variety is its ring of functions $R$. It induces a map on the points which are $\mspec\prs{R}$.

There is however the problem that this lacks functoriality. Images and preimages of maximal ideals aren't necessarily maximal.
For this one looks at the space of prime ideals $\spec\prs{R}$. 
If we look at $R = \CC\brs{x}$ we get $\spec R = \AA^1 \cup \set{\prs{0}}$. We don't have a natural way to add $\prs{0}$ to the topology of $\AA^1$. In higher dimension this is even more problematic.
However, these points have important meaning which we want to understand through the notion of schemes.

\section{Schemes - Basic Definitions}

Let $A \in \Ring$, we define $\spec A$ as the space of prime ideals of $A$.

Given $\aa \ideal A$ we define
\[V\prs{\aa} = \set{\pp \in \spec\prs{A}}{\pp \supseteq \aa} \text{.}\] 
This is in bijection with $\spec\prs{\quot{A}{\aa}}$.

\begin{lemma}
\begin{enumerate}
\item Let $\aa_1, \aa_2 \ideal A$. Then
\[V\prs{\aa_1 \aa_2} = V\prs{\aa_1} \cup V\prs{\aa_2} \text{.}\]
\item If $\set{\aa_i}_{i \in I}$ are ideals of $A$ then
\[V\prs{\sum_{i \in I} \aa_i} = \bigcap_{i \in I} V\prs{\aa_i} \text{.}\]
\item If $\aa_1, \aa_2 \ideal A$ then
\[V\prs{\aa_1} \subseteq V\prs{\aa_2} \iff \sqrt{\aa_1} \supseteq \sqrt{\aa_2} \text{.}\]
\end{enumerate}
\end{lemma}

\begin{proof}
See Hartshorne.
\end{proof}

\begin{remark}
$V\prs{A} = \ns$ and $V\prs{\prs{0}} = \spec A$.
\end{remark}

\begin{definition}[Zariski Topology]
We give $\spec A$ the topology defined by the closed sets $V\prs{\aa}$ for $\aa \ideal A$.

This is called the \stress{Zariski topology}.
\end{definition}

\begin{definition}
Define a sheaf of rings $\Oo = \Oo_A$ on $\spec A$ as follows.
For $U \subseteq \spec A$ open we define
\[\Oo\prs{U} \ceq \set{s \colon U \to \coprod_{\pp \in U} A_\pp}{\substack{\forall \pp \in U \colon s\prs{\pp} \in A_{\pp} \\ \forall \pp \in U \exists \pp \in V \subseteq U \exists a,f \in A \prs{f \notin \pp \, \wedge \, \forall \qq \in V \colon s\prs{\qq} = \frac{a}{\pp} \in A_\qq}}} \text{.}\]
\end{definition}

\begin{proposition}
\begin{itemize}
\item $\Oo\prs{U}$ is a commutative unital ring.
\item If $V \subseteq U$, the map
\begin{align*}
\Oo\prs{U} &\to \Oo\prs{V} \\
s &\mapsto \rest{s}{V}
\end{align*}
is a homomorphism of rings.
\item $\Oo$ is a sheaf on $\spec A$.
\end{itemize}
\end{proposition}

\begin{definition}
If $f \in A$, set \[D\prs{f} \ceq \spec A \setminus V\prs{f} = \set{\pp \in \spec A}{f \notin \pp} \text{.}\]
\end{definition}

\begin{lemma}
The sets $D\prs{F}$ are a basis for the Zariski topology on $\spec A$.
\end{lemma}

\begin{proof}
In Hartshorne.
\end{proof}

\begin{proposition}\label{proposition:spec_properties}
\begin{enumerate}
\item If $\pp \in \spec A$ then the stalk $\Oo_\pp$ of $\Oo$ at $\pp$ is isomorphic to $A_\pp$.
\item $\Oo\prs{\Dd\prs{f}} \cong A_f$.
\item
\[\Gamma\prs{\spec A, \Oo} = \Oo_A\prs{\spec A} = \Oo\prs{D\prs{1}} \cong A_1 \cong A \text{.}\]
\end{enumerate}
\end{proposition}

%1/4/2020

We want to give intuition for the sheaf $\Oo_A$ on $\spec\prs{A}$. Let $U \subseteq X$ open and $s \in \Oo_A\prs{U}$.
Then
\[s \colon U \to \coprod_{\pp \in U} A_\pp \text{.}\]
$A_\pp$ is a local ring with maximal ideal $\pp A_\pp$. Then
\[k_\pp \ceq \quot{A_\pp}{\pp A_\pp}\]
is a field called \stress{the residue field of $A$ at $\pp$}.

From $s \in \Oo\prs{U}$ we get a map
\[\tilde{s} \colon U \to \coprod_{\pp \in U} A_\pp \to \coprod_{\pp \in U} k_\pp \text{.}\]

\begin{example}
Let $A = \ZZ$, we have \[\spec A = \set{\prs{0}, \prs{2}, \prs{3}, \prs{5}, \prs{7}, \ldots}\text{.}\]
Let \[U = D\prs{15} = \spec \ZZ \setminus V\prs{15} \spec\prs \ZZ \setminus \set{\prs{3}, \prs{5}} \text{.}\]

The above proposition says
\[\Oo_{\spec \ZZ}\prs{U} = \ZZ \brs{\frac{1}{15}} \text{.}\]
Take $\frac{11}{5} \in \ZZ\brs{\frac{1}{5}}$.
Thinking of maps of the form $\tilde{s}$ we have maps
\[\prs{p} \mapsto \frac{11}{5} \in \ZZ_\pp \mapsto \frac{11}{5} \in \FF_\pp\]
and
\[\prs{0} \mapsto \frac{11}{5} \in \QQ \mapsto \frac{11}{5} \in \QQ \text{.}\]
\end{example}

\begin{proof}[\ref{proposition:spec_properties}]
\begin{enumerate}
\item Define
\begin{align*}
\Oo_\pp &\xrightarrow{\phi} A_\pp \\
\prs{\pp \in U,s} &\mapsto s\prs{\pp} \text{.}
\end{align*}
\begin{description}
\item[Surjectivity:]
Let $\frac{a}{f} \in A_\pp$, so $a \in A$ and $f \notin \pp$.
Take $U = D\prs{f}$. Then
\begin{align*}
s \colon D\prs{f} &\to \coprod_{\qq \in D\prs{f}} A_\qq \\
\qq &\mapsto \frac{a}{f} \in A_\qq \text{.}
\end{align*}
Then
$\phi\prs{s} = \frac{a}{f}$.
\item[Injectivity:]
Suppose $\pp \in U \subseteq \spec A$ and $s,t \in \Oo\prs{U}$ such that $\phi\prs{s} = \phi\prs{t}$. So $s\prs{\pp} = t\prs{\pp}$.

By definition of $\Oo$ we can shrink $U$ so that
\[s = \frac{a}{p}, \quad t = \frac{b}{g}\]
where $f,g \notin \pp$.

Then
$\frac{a}{f} = \frac{b}{g} \in A_\pp$
so
\[\exists h \notin \pp \colon h\prs{ga - fb} = 0 \in A \text{.}\]

Let \[V = D\prs{f} \cap D\prs{g} \cap D\prs{h}\]
then $\forall \qq \in V \subseteq U \colon S_\qq = t_\qq$.
So
\[\rest{s}{V} = \rest{t}{V}\]
so $s = t \in \Oo_\pp$.
\end{description}
\item
Let
\begin{align*}
\psi \colon A_f &\to \Oo\prs{D\prs{f}} \\
\frac{a}{f} &\mapsto \fcases{D\prs{f} \to \prod_{\qq \ni t} A_\qq \\ \qq \mapsto \frac{a}{f^n} \in \AA_q} \text{.}
\end{align*}
Then $\psi$ is
\begin{description}
\item[Injective:]
$\psi\prs{\frac{a}{f^n}} = \psi\prs{\frac{b}{f^m}}$
hence
$\frac{a}{f^n} = \frac{b}{f^m}$ in $A_\qq$.
Hence
\[\exists h_\pp \notin \pp \colon h_\pp \prs{f^m a - f^n b} = 0 \text{.}\]

Set $\aa = \ann\prs{f^m a - f^n b}$.
So $h_\pp \in \aa \setminus \pp$ hence $\aa \nsubseteq \pp$.
Hence $V\prs{\aa} \cap D\prs{f} = \ns$.
Hence $V\prs{\aa} \subseteq V\prs{f}$
hence $f \in \sqrt{\aa}$ so $f^+ \in \aa$ so $f^+ \prs{f^m a - f^n b} = 0$ so
$\frac{a}{f^n} = \frac{b}{f^m}$ in $A_f$ so $\psi$ is injective.
\item[Surjective:]
We have
\begin{align*}
\psi \colon A_f &\to \Oo\prs{D\prs{f}} \\
\frac{a}{f^n} &\mapsto \brs{\pp \mapsto \frac{a}{f^n}} \text{.}
\end{align*}
We notice two big facts.
\begin{fact}
\[D\prs{f} \subseteq D\prs{g} \iff \exists n \in \NN_+ \colon f^n \in \prs{g}\]
(Exercise / See Hartshorne or Atiyah McDonald).
\end{fact}

\begin{fact}
If $D\prs{f} \subseteq \bigcup_{i \in I} D\prs{f_i}$ then $D\prs{f} \subseteq \bigcup_{j \in J} D\prs{f_j}$ for some $J \subseteq I$ finite.
\end{fact}

Let $s \in \Oo\prs{D\prs{f}}$, we cover $D\prs{f}$ by $\set{V_i}_{i \in I}$ such that $\rest{s}{V_i} = \frac{a_i}{g_i}$ and $\forall \pp \in V_i \colon g_i \notin \pp$.
So $V_i \subseteq D\prs{g_i}$.

Check, using the first fact above, that we may assume that $V_i = D\prs{g_i}$.
Moreover, by the second fact we may assume $I$ is finite.

Note that
\[\frac{a_i}{g_i} = \frac{a_j}{g_j}\]
on $D\prs{g_i} \cap D\prs{g_j} = D\prs{g_i g_j}$, i.e. the equality holds in $A_{g_i g_j}$.
There's $n_{i,j}$ such that
\[\prs{g_i g_j}^{n_{i,j}} \prs{g_j a_i - g_i a_j} = 0\]
and because the index set is finite we may assume $n_{i,j} = n$ for all $i,j$.

Then
\begin{align*}
g_j^{n+1} \prs{g_i^n A_i} - g_i^{n+1} \prs{g_j^n a_j} = 0 \text{.}
\end{align*}
Replace $g_i$ by $g_i^{n+1}$ and $a_i$ by $g_1^n a_i$.
We use $D\prs{g_i} = D\prs{g_i^{n+1}}$.
Then
\[\forall i,j \colon g_j a_i = g_i a_j \text{.}\]

Since $D\prs{g_i}$ cover $D\prs{f}$.
So $f \in \sqrt{\sum_i \prs{g_i}}$ so there's $m \in \NN$ such that $f^m = \sum_i b_i g_i$ for some $b_i \in A$.
Let $a = \sum_i b_i a_i$.
Then
\begin{align*}
g_j a = \sum_i b_i a_i g_j = \sum_i b_i g_i a_j = f^n a_j \text{.}
\end{align*}
Hence
\[\frac{a}{f^m} = \frac{a_i}{g_j}\]
on $D\prs{g_j}$ hence
\[\psi\prs{\frac{a}{f^m}} = s\]
so
$\psi$ is surjective.
\end{description}
\end{enumerate}
\end{proof}

\begin{remark}
There's some kind of analogy between $\spec \ZZ$ and $\spec \CC\brs{x} = \AA^1$.
\end{remark}

\begin{definition}[Ringed Space]
A \stress{ringed space} is a pair $\prs{X, \Oo_X}$ where $X \in \Top$ and $\Oo_X \in \Sh\prs{X, \Ring}$ is a sheaf of rings on $X$.
\end{definition}

\begin{definition}[Morphism of Ringed Spaces]
A morphism $\prs{X, \Oo_X} \to \prs{Y, \Oo_Y}$ of ringed spaces is a pair $\prs{f, f^{\#}}$ where $f \colon X \to Y$ is a continuous map and $f^{\#} \colon \Oo_Y \to f_* \Oo_X$ is a morphism of sheaves on $Y$.
\end{definition}

\begin{definition}[Locally Ringed Space]
A ringed space $\prs{X, \Oo_X}$ is a \stress{locally ringed space} if for all $o \in \X$ the stalk $\Oo_{X,p}$ of $p$ is a local ring.
\end{definition}

\begin{definition}[Local Homomorphism]
Let $R,S \in \Ring$ both local with respective maximal ideals $\mm_R, \mm_S$.

A ring homomorphism
\[\phi \colon R \to S\]
is \stress{local} if $\phi^{-1}\prs{\mm_S} = \mm_R$.
\end{definition}

\begin{definition}[Morphism of Locally Ringed Spaces]
A \stress{morphism of locally ringed spaces} is a morphism
\[\prs{X, \Oo_X} \to \prs{Y, \Oo_Y}\]
is a pair $\prs{f,f^{\#}}$ such that for all $p \in X$,
\[f_p^{\#} \colon \Oo_{Y, f\prs{p}} = \lim_{\substack{\longrightarrow \\ V \ni f\prs{p}}} \Oo_Y\prs{V} \to \lim_{\substack{\longrightarrow \\ V \ni f\prs{p}}} \Oo_X\prs{f^{-1} V}
\to \lim_{p \in U} \Oo_X\prs{U} \to \Oo_{X, p}\]
is a local homomorphism.
\end{definition}

\begin{remark}
$\prs{f, f^{\sharp}}$ is an isomorphism iff $f$ is an homeomorphism and $f^{\#}$ is an isomorphism of sheaves.
\end{remark}

\begin{example}
To understand what $\prs{f_* \Oo_X}_{f\prs{p}}$ is, consider $f\prs{x} = x^2$ where
\[f \colon \spec \CC\brs{t} = \AA^1 \xrightarrow{f} \AA^1 \text{.}\]
Look at the stalk of $f_* \Oo_{\AA^1}$ at the points $x=0$ and $x=1$.
At $x=1$ the map $f$ isn't a local homeomorphism, and indeed the stalk isn't a local ring. It has two maximal ideals. At $x=0$ the stalk is in fact a local ring.
\end{example}

\begin{proposition}
\begin{enumerate}
\item Let $A \in \Ring$. Then $\prs{\spec A, \Oo_A}$ is a locally ringed space.
\item If $\phi \colon A \to B$ is a homeomorphism of rings, then $\phi$ induces a morphism \[\prs{f,f^{\#}} \colon \prs{\spec B, \Oo_B} \to \prs{\spec A, \Oo_A}\]
of locally ringed spaces.
\item Any morphism
\[\prs{\spec B, \Oo_B} \to \prs{\spec A, \Oo_A}\]
of locally ringed spaces is induced from such a ring homomorphism.
\end{enumerate}
\end{proposition}

\begin{proof}
\begin{enumerate}
\item This is true since $\Oo_{A, \pp} \cong A_\pp$ which is a local ring.
\item Given $\phi \colon A \to B$ we define
\begin{align*}
f \colon \spec B &\to \spec A \\
\pp &\mapsto \phi^{-1}\prs{\pp} \text{.}
\end{align*}
For every $\aa \ideal A$ we have \[f^{-1}\prs{V\prs{\aa}} = V\prs{\phi\prs{\aa}} \text{.}\]
Hence $f$ is continuous.

For $V \subseteq \spec\prs{A}$ open define
\begin{align*}
f_V^{\#} \colon \Oo_A\prs{V} &\to \Oo_B\prs{f^{-1}\prs{V}} \\
\set{s \colon V \to \coprod_{\qq \in V} A_\qq} &\mapsto \brs{\pp \mapsto \phi_p \prs{s\prs{f\prs{p}}}}
\end{align*}
where $\phi_\pp \colon A_{\phi^{-1}\prs{\pp}} \to B_\pp$ and by construction $\phi_\pp$ is a local homeomorphism.
The induced map $f_p^{\#}$ is $\phi_p$, so $\prs{f,f^{\#}}$ is a morphism of locally ringed spaces.

\item Given $\prs{f,f^{\#}} \colon \spec B \to \spec A$ where $f^{\#} \colon \Oo_a \to f_* \Oo_B$ we get
\[\phi \colon \Oo_A \prs{\spec A} \to \prs{f_* \Oo_B} \prs{\spec A} \text{.}\]
We get a diagram
\[
\begin{tikzcd}
\phi \colon \Oo_A \prs{\spec A} \arrow[r,"\phi"] \arrow[d] & \prs{f_* \Oo_B} \prs{\spec A} \arrow[d] \\
A_{f\prs{\pp}} \arrow[r, "f_\pp^{\#}", swap] & B_\pp
\end{tikzcd}
\]
where $f_\pp^{\#}$ is local. So
$\phi^{-1}\prs{\pp} = f\prs{\pp}$.

Now the end of the proof follows from the following exercise.

\begin{exercise}
If $\psi_1, \psi_2 \colon F \to G$ are morphisms of sheaves on $X$ and $\psi_{1,p} = \psi_{2,p}$ for all $p \in X$ then $\psi_1 = \psi_2$.
\end{exercise}

\begin{solution}
Consider $\psi = \psi_1 - \psi_2$.
Then the induced morphisms $\psi_P$ on stalks are zero.
Consider
\[0 \to \ker \psi \to F \xrightarrow{\psi} G \text{.}\]
We get for every $P$ the diagram
\[0 \to \prs{\ker \psi}_P \to F_P \xrightarrow{\psi_P = 0} G_P \text{.}\]
Then the middle map is an isomorphism for every $P$, so the inclusion
$\ker \psi \to F$ is an isomorphism, so $F = \ker \psi$ from which $\psi = 0$.
\end{solution}

\stress{Warning:} Let $F,G$ be sheaves on $X$ and consider \[\underline{\hom}\prs{F,G}\prs{U} = \hom\prs{\rest{F}{U}, \rest{G}{U}} \text{.}\]
We have that $\psi$ is a global section of this hom sheaf which is zero on every stalk. There's a map
\[\abhom\prs{F,G}_P \to \hom\prs{F_P, G_P} \text{.}\]
This however isn't generally an isomorphism.
\end{enumerate}
\end{proof}

%2/4/2020

\begin{definition}[Affine Scheme]
An \stress{affine scheme} is a locally ringed space $X, \Oo_X$ which is isomorphic to $\prs{\spec_A, \Oo_A}$ for some $A \in \Ring$.
\end{definition}

\begin{definition}[Scheme]
A \stress{scheme} is a locally ringed space $\prs{X, \Oo_X}$ with an open cover $\prs{U_i}_{i \in I}$ such that $\prs{U_i, \rest{\Oo_X}{U_i}}$ is an affine scheme for all $i \in I$.
\end{definition}

\begin{definition}[Structure Sheaf]
$\Oo_X$ is the \stress{structure sheaf} of the scheme $X$.
\end{definition}

\begin{remark}
We usually write $X$ instead of $\prs{X, \Oo_X}$ and remember there is an associated structure sheaf $\Oo_X$.
\end{remark}

\begin{examples}
\begin{enumerate}
\item
If $k \in \catname{Field}$ then $\spec k = \set{\prs{0}}$ is a scheme
with $\Oo_{\spec k} \prs{\spec k} = k$.
\item If $k \in \catname{Field}$ then
$R \ceq \quot{k\brs{\eps}}{\eps^2} = k + k \eps$ is a local ring with maximal ideal $\mm \ceq \prs{\eps} = k \eps$.
Then 
\[X \ceq \spec \prs{\quot{k\brs{\eps}}{\eps^2}} = \set{\mm}\]
is a scheme $\Oo_X\prs{X} = R$.

The map
\begin{align*}
R &\to k \\
\eps &\mapsto 0
\end{align*}
induces a morphism of schemes
\[\spec k \rmono \spec \quot{k\brs{\eps}}{\eps^2} \text{.}\]
This is an homeomorphism but not an isomorphism of schemes.

The intuition is that $\spec k$ is a point and $X$ is a fattened point, as it's in some sense given by looking at both the point an its tangent space.

\item Let $p$ a prime number and let \[X \ceq \spec \ZZ_{\prs{p}} = \set{\frac{a}{b}}{\substack{a,b \in \ZZ \\ b \notin p \ZZ}}\text{.}\]
Then $X = \set{p \ZZ_{\prs{p}}, \prs{0}}$.
The point $p \ZZ_{\prs{p}}$ is closed point (in the case of affine schemes this is equivalent to maximality).
The point $\prs{0}$ is open and dense (it's not closed, and $\abs{X} = 2$).
\begin{itemize}
\item
We have
\[\Oo_X\prs{\prs{0}} \cong \Oo_{X, p \ZZ_{\prs{p}}} \cong \Oo_X\prs{D\prs{p}} \cong \prs{\ZZ_{\prs{p}}}_p \cong \QQ \text{.}\]
\item
The residue field at $p \ZZ_{\prs{p}}$ is $\FF_p$.

The residue field at $\prs{0}$ is $\QQ$.
\end{itemize}
\item
Define $\AA_k^1 \ceq \spec k\brs{x}$.
Then
\[\AA^1_k = \set{\prs{0}} \cup \set{\prs{f\prs{x}}}{\text{$f$ is irreducible}} \text{.}\]

The ideals coming from irreducible polynomials are maximal ideals, hence closed points.
The point $\prs{0}$ is dense in $\AA_k^1$ and is called \stress{the generic point}.

If $k = \bar{k}$ then the closed points are in bijection with the field $k$ as they come from polynomials of the form $x - \alpha$.
Otherwise, the closed points are in bijection with monic irreducible polynomials, which we can think of as the $\gal\prs{\quot{\bar{k}}{k}}$ orbits of elements in $k$.

\item 
Let $\AA_k^n = \spec\prs{k\brs{x_1, \ldots, x_n}}$.
If $k = \bar{k}$ we have
\[\AA_k^2 = \set{\prs{0}} \cup \set{\prs{x_1 - t_1, x_2 - t_2}}{t_1, t_2 \in k} \cup \set{\prs{f\prs{x,y}}}{\text{$f$ is irreducible}} \cup \ldots \text{.}\]
The point $\prs{0}$ is the generic point which is here not dense or open. The points in the second set are closed and the points in the third set are with closure containing all closed points $\prs{t_1, t_2}$ such that $f\prs{t_1, t_2} = 0$.
\end{enumerate}
\end{examples}

\begin{remark}
If $F$ is a sheaf on $X$ then $F\prs{ns} = 0$.

It follows from covering the empty set by the empty open cover and using the sheaf axioms.
\end{remark}

\begin{definition}[Disjoint Union of Schemes]
Let $X,Y$ be schemes, and let $i_X \colon X \rmono X \amalg Y$ and $i_Y \colon Y \rmono X \amalg Y$ we define $X \amalg Y$ to be $\prs{X \amalg Y, \prs{i_{X}}_*\Oo_X \oplus \prs{i_Y}_* \Oo_Y}$.
\end{definition}

\begin{definition}[Gluing of Schemes]
Let $X,Y$ be schemes and let $U \subseteq X, V \subseteq Y$ be open sets.
Suppose
\[\phi \colon \prs{U, \rest{\Oo_X}{U}} \riso \prs{V, \rest{\Oo_Y}{V}}\]
is an isomorphism of locally ringed spaces.
Suppose $i \colon X \rmono Z$ and $j \colon Y \rmono Z$ be topological inclusions.

We define the \stress{gluing} $Z = \quot{X \amalg Y}{\phi}$
to be
\[\prs{\quot{X \amalg Y}{\substack{\forall u \in U \\ u \sim \phi\prs{u}}}, \Oo_Z}\]
where for $W \subseteq Z$,
\[\Oo_Z\prs{W} = \set{\prs{s, s'}}{\substack{s \in \Oo_X\prs{i^{-1} W} \\ s' \in \Oo_Y\prs{j^{-1} W}} \text{ such that $\phi\prs{\rest{s}{U \cap i^{-1} W}} = \rest{s'}{V \cap j^{-1} W}$}} \text{.}\]
This is a scheme with scheme maps $X \rmono Z$ and $Y \rmono Z$.
\end{definition}

\begin{remark}
One can interpret the gluing as a special case of a pushout in the category of schemes.
\end{remark}

\begin{example}
We can glue $\AA^1$ to itself along $\AA^1 \setminus \set{0}$ using the identity on $\AA^1 \setminus \set{0}$ as the gluing map $\phi$.

We get a line with a double point at $0$.
\end{example}

\begin{example}
We can glue $\AA_k^1$ to $\AA_k^1$ by identifying $\AA^1 \setminus \set{0}$ with \[\AA^1 \setminus \set{0} = \spec k\brs{x, \frac{1}{x}} \subseteq \AA^1_k\]
under $x \mapsto \frac{1}{x}$.

We get $\PP_k^1$.
\end{example}

Let $S$ be an $\NN$-graded ring, so $S = \bigoplus_{d \geq 0} S_d$ with each $S_d$ an abelian group such that $S_d \cdot S_e \subseteq S_{d + e}$.

\begin{definition}[Homogeneous Element]
An element $s \in S$ is \stress{homogeneous} if $s \in S_d$ for some $d \in \NN$.
\end{definition}

\begin{example}
Let $k \in \catname{Field}$. Then $S = k\brs{X}$ is graded with $S_d = k x^d$.

$S' = k\brs{x,y,z}$ is graded with $S_d$ the set of monomials of total degree $d$.
\end{example}

\begin{definition}[Homogeneous Ideal]
An ideal $\aa \subseteq S$ is \stress{homogeneous} if
\[\aa = \bigoplus_{d \geq 0} \prs{\aa \cap S_d} \text{.}\]
\end{definition}

\begin{fact}
An homogeneous ideal is equivalently an ideal generated by homogeneous elements.
\end{fact}

\begin{notation}
Denote \[S_+ \ceq \bigoplus_{d \geq 1} S_d \text{.}\]
\end{notation}

\begin{definition}
Let $S$ be an $\NN$-graded ring, we define
\[\mrm{Proj} S \ceq \set{\pp \in S}{\text{$\pp$ is a homogeneous prime ideal not containing $S_+$}} \text{.}\]
\end{definition}

\begin{definition}
For any $\aa \ideal S$ homogeneous let
\[V\prs{\aa} = \set{\pp \in \mrm{Proj}S}{\pp \supseteq \aa} \text{.}\]
\end{definition}

\begin{lemma}
\begin{align*}
V\prs{\aa_1 \aa_2} &= V\prs{\aa_1} \cup V\prs{\aa_2} \\
V\prs{\sum_{i \in I} \aa_i} = \bigcap_{i \in I} V\prs{\aa_i}
\end{align*}
\end{lemma}

\begin{definition}
We put the Zariski topology on $\mrm{Proj}S$ such that $Z \subseteq \mrm{Proj}S$ is closed iff $Z = V\prs{aa}$ for some homogeneous ideal $\aa \subseteq S$.
\end{definition}

\begin{definition}[$\Oo_{\mrm{Proj} S}$]
For $\pp \in \Proj S$ let
\[T_\pp \ceq \set{s \in S}{\substack{\text{$s$ is homogeneous} \\ s \notin \pp}} \text{.}\]
Let $T_\pp^{-1} S$ the localisation of $S$ at $T_\pp$.

We define
\[S_{\prs{p}} = \set{s \in T_\pp^{-1} S}{\text{$s$ has degree }} \text{.}\]

Define a sheaf $\Oo_{\Proj S}$ on $\Proj S$ as follows.
For $U \subseteq \Proj S$ let
\[\Oo_{\Proj S} \ceq \set{s \colon U \to \coprod_{\pp \in U} S_{\prs{\pp}}}{\substack{\forall \pp \colon s\prs{\pp} \in S_{\prs{\pp}} \\ \star}}\]
where condition $\star$ means that for all $\pp \in U$ there's a neighbourhood $V \subseteq U$ of $\pp$ and there are homogeneous $a,f \in S$ such that for all $\aa \in V$ it holds that $f \notin \aa$ and
\[s\prs{\qq} = \frac{a}{f} \in S_{\prs{\aa}} \text{.}\]
\end{definition}

\begin{definition}
For any $f \in S_+$ we define \[D_+\prs{f} = \set{\pp \in \Proj S}{f \notin \pp} \text{.}\]
\end{definition}

\begin{remark}
The set $D_+\prs{f}$ is open in $\Proj S$.
\end{remark}

\begin{proposition}
Denote $\Oo = \Oo_{\Proj S}$.
\begin{enumerate}
\item For any $\pp \in \Proj S$ the stalk $\Oo_\pp$ of $\Oo$ at $\Pp$ is isomorphic to $S_{\prs{\pp}}$.
\item
The sets $D_+\prs{f}$ for $f \in S_+$ cover $\Proj S$.
\item
\[\prs{D_+\prs{f}, \rest{\Oo}{D_+\prs{f}}} \cong \spec S_{\prs{f}}\]
as locally-ringed spaces where $S_{\prs{f}}$ is defined to be the subring of $S_f$ of degree $0$ elements.
\end{enumerate}
\end{proposition}

\begin{proof}
See Hartshorne.
\end{proof}

\begin{corollary}[Projective Space over $A$]
$\Proj S$ is a scheme.
\end{corollary}

\begin{definition}
For any $A \in \Ring$ we can define
\[\PP_A^n \ceq \Proj A\brs{x_0, \ldots, x_n}\]
with the grading described in the proposition.

This is called \stress{the projective space over $A$}.
\end{definition}

\begin{definition}[Scheme over a Scheme]
Let $S$ be a scheme. \stress{A scheme over $S$} is a scheme $X$ together with a map $X \to S$.
\end{definition}

\begin{definition}[]
Let $X \xrightarrow{\pi_1} S, Y \xrightarrow{\pi_2} S$ be schemes over $S$.
A morphism between $\pi_1, \pi_2$ is $\phi \colon X \to Y$ such that
the diagram
\[
\begin{tikzcd}
X \arrow[rr, "\phi"] \arrow[dr, "\pi_1", swap] & & Y \arrow[dl, "\pi_2"] \\
& S & 
\end{tikzcd}
\]
commutes.
\end{definition}

\begin{notation}
Denote by $\quot{\Sch}{S}$ for the category of schemes over a scheme $S$.

Let $A \in \Ring$, we denote by $\quot{\Sch}{A}$ for the category of schemes over $\spec A$.
\end{notation}

\begin{proposition}
Let $k \in \catname{Field}$ such that $k = \bar{k}$. Then there's a fully-faithful functor $t \colon \quot{\catname{Var}}{k} \to \quot{\Sch}{k}$ from the category of varieties over $k$ to $\quot{\Sch}{k}$.

For $V \in \quot{\catname{Var}}{k}$, $V$ is homeomorphic to the set of closed points of $t\prs{V}$ and the sheaf $\Oo_V$ of regular functions on $V$ is isomorphic to $\rest{\Oo_{t\prs{V}}}{V}$.
\end{proposition}

\begin{proof}
Define
\[t\prs{V} = \set{\text{non-empty irreducible closed subsets of $V$}} \text{.}\]

See Hartshorne for details.
\end{proof}

\begin{example}
Let $f\prs{x} = x^5 + 1$. Then the zero locus of $y^2 - f\prs{x}$ in $\AA_\CC^2$ is not compact in some sense.
We can homogenise this to
$$\brs{0:1:0} \in \set{z^3 y^2 - x^5 - z^5 = 0} \subseteq \PP_\CC^2 \text{.}$$
The first curve is smooth, but the second one isn't smooth of $\brs{0:1:0}$.
However, this can be embedded into a smooth curve.

We have
$$\spec\prs{\quot{\CC\brs{x,y}}{y^2 - f\prs{x}}} \subseteq \spec \CC\prs{x,y} = \AA_\CC^2 \text{.}$$
Then
$$\Proj\prs{\quot{\CC\brs{x,y,z}}{y^2 - z^6 f\prs{\frac{x}{z}}}} \text{,}$$
by considering $\deg\prs{x} = \deg\prs{z} = 1$ and $\deg\prs{y} = 3$ can be seen as a graded algebra.
We have $$\spec\prs{\quot{\CC\brs{x,y}}{y^2 - f\prs{x}}} = D\prs{Z}$$ inside this, and that this projective variety is smooth.
\end{example}

\subsection{Properties of Schemes:}

Let $X \in \Sch$.

\begin{definition}[Connected Sceheme]
$X$ is \stress{connected} if its underlying topological space is connected.
\end{definition}

\begin{definition}[Irreducible Sceheme]
$X$ is \stress{irreducible} if its underlying topological space is irreducible.
\end{definition}

\begin{definition}[Reduced Scheme]
$X$ is \stress{reduced} if $\OOO_X\prs{U}$ is reduced for every open $U \subseteq X$.
\end{definition}

\begin{exercise}
$X$ is reduced iff $\OOO_{X,P}$ is reduced for all $P \in X$.
\end{exercise}

\begin{definition}[Integral Scheme]
$X$ is \stress{integral} if $\OOO_X\prs{U}$ is an integral domain for every open $U \subseteq X$.
\end{definition}

\begin{example}
Let $X = \spec A$ for some $A \in \catname{Ring}$.
\begin{itemize}
    \item $X$ is irreducible iff $\nil\prs{A}$ is prime.
    \item $X$ is reduced iff $\nil A = 0$.
    \item $X$ is integral iff $A$ is an integral domain.
\end{itemize}
\end{example}

\begin{proposition}
A scheme $X$ is integral iff it's reduced and irreducible.
\end{proposition}

\begin{proof}
Assume $X$ is integral. Then there are no nilpotent elements so $X$ is reduced.
Assume towards a contradiction $X$ isn't irreducible, there are $U_1, U_2 \subseteq X$ open such that $U_1 \cap U_2 = \ns$.
Now, $\OOO_X\prs{U_1 \cap U_2} = \OOO_X\prs{\ns} = 0$, so
$$\OOO_X\prs{U_1 \cup U_2} = \OOO_X\prs{U_1} \oplus \OOO_X\prs{U_2}$$
which isn't an integral domain.
Hence $X$ is irreducible.

Conversely, assume $X$ is reduced and irreducible. Let $U \subseteq X$ open and $f,g \in \OOO_X\prs{U}$ nonzero such that $fg = 0$.
Let $$Y \ceq \set{x \in U}{f_x \in \mm_x \ideal \OOO_{X,x}}$$
and
$$Z \ceq \set{x \in U}{g_x \in \mm_x \ideal \OOO_{X,x}} \text{.}$$
These are closed in $U$, and $Y \cup Z = U$.
Now, $X$ is irreducible so $U$ is irreducible so $Y = U$ or $Z = U$ and assume WLOS $Y=U$.
If $Y = \spec A \subseteq U$ then $\rest{f}{V} \in \nil_A = 0$ so $f = 0$. Otherwise, we vcover $Y$ by affines and get the same result.
Now $\OOO_X\prs{U}$ is an integral domain.
\end{proof}

\begin{definition}[Locally Noetherian Scheme]
$X$ is \stress{locally Noetherian} if it has an over cover $\set{\spec A_i}_{i \in I}$ with $A_i$ noetherian for every $i \in I$.
\end{definition}

\begin{definition}[Quasi-Compact Scheme]
$X$ is \stress{quasi-compact} if any open cover of $X$ has a finite subcover.
\end{definition}

\begin{definition}[Quasi-Compact Morphism]
A morphism $f \colon X \to Y$ is \stress{quasi-compact} if there's an affine open cover $\prs{V_i}_{i \in I}$ of $Y$ such that $f^{-1}\prs{V_i}$ is quasi-compact for all $i \in I$.
\end{definition}

\begin{definition}[Noetherian Scheme]
$X$ is \stress{Noetherian} if it's locally Noetherian and quasi-compact.
\end{definition}

\begin{remark}
$X$ is Noetherian iff there's a finite cover of $X$ by affine schemes $\spec A_i$ with $A_i$ Noetherian.
\end{remark}

\begin{lemma}
Let $\spec A$ and $\spec B$ be open affine schemes in $X$. Then $\spec A \cap \spec B$ is covered by open sets that are distinguished (i.e. of the form $D\prs{f}$) in both $\spec A$ and $\spec B$.
\end{lemma}

\begin{proof}
Let $P \in \spec A \cap \spec B$. $\spec A \cap \spec B$ is open in $A$ so there's $f \in A$ such that $$P \in \spec A_f \subseteq \spec A \cap \spec B\text{.}$$
Now in $\spec A_f$ which is open in $\spec B$ we can get $\spec B_g$ similarly.

Denote by $g'$ the map of $g$ in $\OOO_X\prs{\spec A_f}$ and write $g' = \frac{\tilde{g}}{f^k}$.
Then $$\spec B_g \subseteq \spec A_f \setminus \set{Q}{g' \in Q} = \spec\prs{A_f}_{g'} = \spec \prs{A_{f \tilde{g}}} \text{.}$$
\end{proof}

\begin{lemma}[Affine Locality Test]
Let $\star$ be a property of affine subschemes of $X$ such that the following holds.
\begin{enumerate}
    \item If $\spec A \rmono X$ has $\star$ then so does $\spec A_f$ for all $f \in A$.
    \item If $\prs{f_1, \ldots, f_n} = A$ and $\spec A_{f_i}$ has $\star$, so does $\spec A$.
\end{enumerate}
Assume $X = \bigcup_{i \in I} A_i$ and that every $A_i$ has $\star$. Then every open affine $\spec B \rmono X$ has $\star$.
\end{lemma}

\begin{proof}
Let $\spec B \rmono X = \bigcup_{i \in I} \spec A_i$ when $A_i$ has $\star$ for every $i \in I$.
Cover $\spec B$ with finitely many opens $\spec B_{g_j}$ which are simultaneously distinguished in $\spec B$ and $\spec A_i$ for some $i \in I$. This is possible by the previous lemma.

Then $\spec B_{g_j}$ has $\star$. Note $\prs{g_1, \ldots, g_n} = B$, so $\spec B$ has $\star$ by the assumptions on $\star$.
\end{proof}

\begin{proposition}
$X$ is locally Noetherian iff for every oepn $\spec A \subseteq X$, $A$ is Noetherian.

In particular, $\spec A$ is a Noetherian scheme iff $A$ is Noetherian.
\end{proposition}

\begin{proof}
We need to show the properties of the affine locality test.

For 1, if $A$ is Noetherian so is $A_f$.
For 2, we prove a lemma.
\begin{lemma}
If $\aa \ideal A$ and $f_i \in A$ are such that $\prs{f_1, \ldots, f_n} = A$ with localisation maps $\phi_i \colon A \to A_{f_i}$, then
$$\aa = \bigcap_{i \in [n]} \phi_i^{-1}\prs{\phi_i\prs{\aa} \cdot A_{f_i}} \text{.}$$
\end{lemma}
\begin{proof}
$\subseteq$ is clear, so we prove the other direction. Let $b$ in the RHS and write $\phi_i\prs{b} = \frac{a_i}{f_i^k}$ with $a_i \in \aa$ and $k > 0$.
Then
$$f_i^m\prs{f_i^k b - a_i} = 0 \text{.}$$
Then
$$f_i^{m+k b \in \aa} \text{.}$$
Note that $\prs{f_1^{m+k}, \ldots, f_n^{m+k}} = A$, because $I=A \iff \sqrt{I} = A$, or because $D\prs{f^n} = D\prs{f}$.
So, $1 = \sum_{i \in [n]}c_i f_i^{k+m}$ for $c_i \in A$.
Then $$b = \sum_{i \in [n]} c_i f_i^{k+m}b \in \aa \text{.}$$
\end{proof}

We want to show $A$ is Noetherian assuming $A_{f_i}$ are.
If $$\aa_1 \subseteq \aa_2 \subseteq \aa_3 \subseteq \ldots$$
is an ascending chain of ideals in $A$ then
$$\phi_i\prs{\aa_1} \subseteq \phi\prs{\aa_2} \subseteq \ldots$$
is an ascending chain in $A_{f_i}$, so it stabilises.
Since there are finitely many $f_i$ and since $\aa = \bigcap_{i \in [n]} \phi_i^{-1}\prs{\phi_i\prs{\aa_i}}$, the $\aa_j$ must stabilise.
\end{proof}

\begin{definition}[Morphism Locally of Finite Type]
A morphism of $f \colon X \to Y$ of schemes is \stress{locally of finite type} if there's an open cover $\prs{V_i}_{i \in I}$ of $Y$ with $V_i \cong \spec B_i$ such that
$$f^{-1}\prs{V_i} = \bigcup_{j \in J_i} \spec A_{i_j}$$
such that $A_{i_j}$ is a finitely-generated $B_i$-agebra for every $i \in I$.
\end{definition}

\begin{definition}[Morphism of Finite Type]
A morphism $f \colon X \to Y$ of schemes is \stress{of finite type} if we can take the $J_i$ above to be finite.
\end{definition}

\begin{definition}[Finite Morphism]
A morphism $f \colon X \to Y$ of schemes is \stress{finite} if there's an open cover $\prs{V_i}_{i \in I}$ of $Y$ with $V_i \cong \spec B_i$ and such that $f^{-1}\prs{V_i} \cong \spec A_i$ with $A_i$ a finite $B_i$-module.
\end{definition}

\begin{definition}[Open Immersion]
Let $X$ be a scheme and $U \subseteq X$ open. Then $\prs{U, \rest{\OOO_X}{U}}$ is a subscheme. Then the map $U \rmono X$ is an \stress{open immersion}.
\end{definition}

\begin{definition}[Closed Immersion]
A \stress{closed immersion} is a morphism $f \colon Y \to X$ such that the following holds.
\begin{enumerate}
    \item $f$ gives a homeomorphism onto a closed subset of $X$.
    \item $f^{\#} \colon \OOO_X \to f_*\OOO_Y$ is surjective.
\end{enumerate}
\end{definition}

\begin{definition}[Closed Subscheme]
A \stress{closed subscheme} is an equivalence class of such morphisms.
\end{definition}

\begin{remark}
Why can't we do what we did for an open immersion?
Look at
\begin{align*}
    \spec \CC &\rmono \spec \CC\brs{t} \\
    \CC\brs{x} &\to 0 \\
    t &\mapsto 0 \text{.}
\end{align*}
Let $i \colon \CC \rmono \spec \CC\brs{t}$, and $X = \spec \CC\brs{t}$.
Then
$i^{-1}\OOO_X = \OOO_{X,x}$.
\end{remark}

\begin{remark}
A map $A \to \quot{A}{I}$ induced a closed immersion $\spec \prs{\quot{A}{I}} \rmono \spec \prs{A}$.
\end{remark}

\begin{definition}
Let $Y \subseteq X$ a closed subset of a scheme $X$.
If $X = \spec A$ set $I = \bigcap_{P \in Y} P \ideal A$.
Then $V\prs{I} = Y$. The induced closed subscheme structure is given by
$$Y = \spec\prs{\quot{A}{I}} \rmono \spec \prs{A} \text{.}$$
\end{definition}

\begin{remark}
$\spec\prs{\quot{A}{I}}$ is indeed reduced.
\end{remark}

In general, to define the induced closed subscheme structure, we cover $X$ with open affines $U_i$ and set $Y_i = Y \cap U_i$.
We get closed subschemes $Y_i \rmono U_i$. We glue these to obtain a sheaf on $Y$.
To construct these we need to give isomorphisms
$$\rest{\OOO_{Y_i}}{Y_i \cap Y_j} \xrightarrow{\phi_{i,j}} \rest{\OOO_{Y_j}}{Y_i \cap Y_J}$$
such that
$$\rest{\phi_{i,k}}{Y_{i,j,k}} = \rest{\phi_{j,k}}{Y_{i,j,k}} = \rest{\phi_{i,j}}{Y_{i,j,k}}$$
with the notation $Y_{i,j,k} = Y_i \cap Y_j \cap Y_k$.

\begin{definition}[Dimension]
Let $X \in \Sch$. The \stress{dimension} of $X$ is the supremum of the length of any chain $$Z_0 \subsetneq Z_1 \subsetneq Z_2 \subsetneq \ldots$$ of closed irreducible subsets of $X$.
\end{definition}

\begin{remark}
If $X = \spec A$, then $\dim X$ is the Krull dimension of $A$.
\end{remark}

\begin{proposition}[Fibre Products]\label{proposition:fibre_products}
Fibre products exist in $\Sch$.

Let $X \to S$ and $Y \to S$ be morphisms of schemes. There exists \stress{the fibre product} $X \times_S Y$ together with maps $p_1 \colon X \times_S Y \to X$ and $p_2 \colon X \times_S Y \to Y$ such that for every $S$-scheme $Z$ and any pair of morphisms $Z \to X$ and $Z \to Y$ there is a unique morphism $Z \to X \times_S Y$ such that the following diagram commutes.
$$
\begin{tikzcd}
Z \arrow[dr, dotted, "\phi"] & & \\ & X \times_S Y \arrow[r, "p_2"] \arrow[d, "p_1", swap] & Y \arrow[d] \\ & X \arrow[r] & S
\end{tikzcd}
$$
\end{proposition}

\begin{remark}
The fibre product when it exists is unique up to a unique isomorphism. We shall use this during the proof when moving from affine schemes to general ones.
\end{remark}

\begin{lemma}\label{lemma:fibre_products}
Let $X,Y \in \Sch\prs{S}$ with respective maps $f,g$ to $S$ such that $X \times_S Y$ exists with maps $p_1$ to $X$ and $p_2$ to $Y$. Let $U \subseteq X, V \subseteq Y, T \subseteq S$ open such that $f\prs{U}, g\prs{V} \subseteq T$. Then
$$F \ceq p_1^{-1}\prs{U} \cap p_2^{-1}\prs{V} \cong U \times_T V \text{.}$$
\end{lemma}

\begin{proof}
View the following diagram.
$$
\begin{tikzcd}
Z \arrow[r] \arrow[d] & V \arrow[d] \\ U \arrow[r] & T
\end{tikzcd}
$$
we extend it to a diagram
$$
\begin{tikzcd}
Z \arrow[r] \arrow[d] & V \arrow[d] \arrow[r, hookrightarrow] & Y \arrow[dd] \\ U \arrow[d,hookrightarrow] \arrow[r] & T \arrow[dr, hookrightarrow] & \\
X \arrow[rr] & & S
\end{tikzcd}
$$
then by diagram chase the maps from $Z$ factor through $p_1^{-1}\prs{U} \cap p_2^{-1}\prs{V}$.
\end{proof}

\begin{proof}
\begin{description}
\item[Affine Case:]
Assume first that $S = \spec A, X = \spec B, Y = \spec C$. We show $\spec \prs{B \times_A C}$ is the tensor product $X \times_S Y$.
This follows from exercise 2.4 in Hartshorne and the fact that the cofibre product of rings is the tensor product. We explain that explicitly.

Recall
$$\hom\prs{B \otimes_A C, R} \leftrightarrow \hom\prs{B, R} \times_{\hom A,R} \hom\prs{C,R} \text{.}$$
For any scheme $Z$ and any ring $A$ we have
$$\hom\prs{Z, \spec A} \leftrightarrow \hom\prs{A, \Gamma\prs{Z, \OOO_Z}}$$
so
\begin{align*}
    \hom\prs{Z, \spec\prs{B \otimes_A C}} \to \hom\prs{B \otimes_A C, R} &\leftrightarrow \hom\prs{B, R} \times_{\hom\prs{A,R}} \hom\prs{C, R} \\&\leftrightarrow \hom\prs{Z, \spec B} \times_{\hom\prs{Z, \spec A}} \hom\prs{Z, \spec C}
\end{align*}
where the arrows are natural bijections.
\item[If $S$ is affine:]
We apply \ref{lemma:fibre_products}. If $S$ is affine, cover $X,Y$ with open affines $\prs{U_i}_{i \in I}, \prs{V_j}_{j \in J}$ and then glue $U_i \times_S V_j$.

We need to give isomorphisms on the overlaps. By the lemma $\prs{U_i \cap U_{i'}} \times_S \prs{V_j \cap V_{j'}}$ exists and is unique up to a a unique isomorphism.
Then uniqueness implies $\phi_{i,j} = \phi_{j,k} \circ \phi_{i,j}$, as both sides satisfy the universal property.

Then we glue to get $X \times_S Y$.

\item[General Case:]
Cover $S$ with $\prs{W_i}_{i \in I}$ affine. By the lemma we get the fibre products $f^{-1}\prs{W_i} \times_{W_i} g^{-1}\prs{W_i}$. Gluing these we get the result.
\end{description}
\end{proof}

\begin{example}
Let $X \in \Sch$, then $X \times_X X \cong X$.

Specifically $$\spec k \times_{\spec k} \spec k \cong \spec k \text{.}$$
\end{example}

\begin{example}
Examine $\spec \CC \times_{\spec \RR} \spec \CC$ under the inclusions $\RR \rmono \CC$ which induce $\spec \CC \to \spec \RR$. This is $\spec \prs{\CC \otimes_{\RR} \CC}$. This is $\spec\prs{\CC \times \CC}$.
There's an isomorphism
\begin{align*}
    \C \otimes_\RR \CC &\riso \CC \times \CC \\
    z \otimes 1 &\mapsto \prs{z, \bar{z}} \\
    z \otimes t &\mapsto \prs{zt, \bar{z} t} \text{.}
\end{align*}
Now, $\spec\prs{\CC \times \CC}$ is $\spec\prs{\CC} \amalg \spec\prs{\CC}$ by exercise 2.4 and by the disjoint union being the coproduct of schemes.
\end{example}

\begin{example}
Examine the following diagram.
$$
\begin{tikzcd}
& \overline{\FF_p} & \\
\FF_{p^2} \arrow[ur] & & \FF_{p^2} \arrow[ul] \\
& \FF_p \arrow[ur] \arrow[ul] &
\end{tikzcd}
$$
Now
\begin{align*}
    \spec \FF_{p^3} \times_{\spec \FF_p} \spec \FF_{p^2} &\cong \spec\prs{\FF_{p^3} \otimes_{\FF_p} \FF_{p^3}} \cong \spec\prs{\FF_{p^6}} \text{.}
\end{align*}
For the second isomorphism, we define
\begin{align*}
    \FF_{p^3} \otimes \FF_{p^2} &\to \FF_{p^6} \\
    a \otimes b &\mapsto ab \text{.}
\end{align*}
This is clearly surjective and is a map between vector spaces of the same dimension and is therefore an isomorphism.
\end{example}

\begin{definition}[Fibres]
Let $f \colon X \to Y$ be a morphism of shcemes and let $y \in Y$. Let $k_y = \quot{\OOO_{Y,y}}{\mm_y}$. Then
there's an induces diagram
$$
\begin{tikzcd}
X_y \arrow[r] \arrow[d] & X \arrow[d, "f"] \\ \spec k_y \arrow[r, hookrightarrow] & Y
\end{tikzcd}
$$
where $X_y \ceq \spec k_y \times_Y X$ is \stress{the fibre of $f$ above $y$}.
\end{definition}


\begin{exercise}
$X_y$ is homeomorphic to $f^{-1}\prs{y}$.
\end{exercise}

\begin{example}
Let $R \ceq \quot{\ZZ\brs{x,y}}{y^2 - x^3 + 15}$, then $\spec R$ has a natural map to $\spec \ZZ$ given by the map $\ZZ \to \ZZ\brs{x,y} \to R$.
There is a closed immersion $\spec \FF_p \rmono \spec \ZZ$ given by the morphism $\ZZ \to \FF_p$.
We get the the fibre of $f$ at $\prs{p}$ is $\spec \prs{\quot{\FF_p\brs{x,y}}{\prs{y^2 - x^3 + 15}}}$.

Looking at dimensions, we have $\dim \ZZ = 1$, $\dim R = 2$, $\dim \FF_p = 0$ and $\dim \spec \prs{\quot{\FF_p\brs{x,y}}{\prs{y^2 - x^3 + 15}}} = 1$.
\end{example}

\begin{example}
Let $k = \CC$ and examine $\spec R$ where $R \ceq \quot{k\brs{x,y,t}}{\prs{ty - x^2}}$.
 We map $\spec R \to \spec k\brs{t} \cong \AA^1_\CC$. The closed points of $\AA^1_\CC$ are in correspondence with elements of $\CC$. So we get a map $\spec k \rmono \spec k\brs{t}$ coming from the map
 \begin{align*}
     k\brs{t} &\to k \\
     t &\mapsto a \text{.}
 \end{align*}
 We get that the fibre at $a$ is $\spec\prs{\quot{k\brs{x,y}}{\prs{ay - x^2}}}$, as in the following diagram.
 $$
 \begin{tikzcd}
 \spec\prs{\quot{k\brs{x,y}}{\prs{ay - x^2}}} \arrow[d] \arrow[r, hookrightarrow] & \spec R \arrow[d, "f"] \\ \spec k \arrow[r, hookrightarrow] & \spec k\brs{t}
 \end{tikzcd}
$$
Denote $X = \spec R$ and $X_a$ the fibre. Then at $a=0$ we have $X_0 \cong \spec \prs{\quot{k\brs{x,y}}{x^2}}$.

We say that the parabolas ``degenerate'' into the double line.

We can also take the fibre at the generic point $\eta \colon \spec k\prs{t} \rmono \AA_\CC^1$. The fibre $X_\eta$ is then $\spec \prs{\quot{k\prs{t}\brs{x,y}}{\prs{ty - x^2}}}$. This is a conic over the field $k\prs{t}$.
\end{example}

\begin{remark}
Many questions in algebraic geometry amount to asking what we can learn about $X_a$ from the \stress{generic fibre} $X_\eta$, in the above notation.
\end{remark}

\begin{example}
Take $k = \CC$.
We can take the same example over $\spec k\brs{\brs{x}}$. We have $\spec k\brs{\brs{t}} = \set{\prs{0}, \prs{t}}$. Then there are two points, $t=0$ and the generic point.
\end{example}

\begin{example}
We can examine the first example over $\spec \ZZ_p$ or $\spec \ZZ_{\prs{p}} = \set{\prs{0}, \prs{p}}$.
\end{example}

\begin{definition}[Base Change]
Let $S \in \Sch$ and $X,S' \in \Sch\prs{S}$. Then
$X \times_S S'$ is an $S'$-scheme.
This is called \stress{the base change of $X$ to $S'$}.
\end{definition}

\begin{example}
Let $X = \spec \RR\brs{x}$ which maps to $\spec \RR$. We have also a morphism $\spec \CC \to \spec \RR$, and the fibre product of these maps is $\spec \CC\brs{x} \cong \AA_\CC^1$.
\end{example}

\begin{remark}
Base change is transitive. If $X,S',S'' \in \Sch\prs{S}$, then
$$X \times_{S} S'' \cong \prs{X \times_S S'} \times_{S'} S'' \text{.}$$
This can be checked by the universal property.
\end{remark}

\begin{definition}[Functor of Points]
Let $X \in \Sch$. We get a functor
$$\Sch^{\op} \to \Set$$
which sends $S$ to $\hom \prs{S,X}$ and a morphism $g \colon S \to S'$ to post-composition $f \to g \circ f$.
\end{definition}

\begin{remark}
\begin{itemize}
    \item
    By Yoneda's lemma, the functor $\hom\prs{-,X}$ determines the scheme.
    \item
    It's enough to work with affine schemes to understand this.
    \item
    Most interesting schemes are best understood in terms of their functor of points.
\end{itemize}
\end{remark}

\begin{notation}
Denote $X\prs{S} \ceq \hom\prs{S,X}$.
\end{notation}

\begin{example}
We define $M_g$ the \stress{moduli functor of (smooth, projective) curves of genus $g$}.

One can define genus generally such that the genus of a complex curve is its topological genus as a real manifold.
We define
$$M_g\prs{S} = \quot{\set{\text{relative curves $C \to S$ of genus $g$}}}{\sim}$$
where a \stress{relative curve} $C \to S$ is a morphism such that for any $k = \bar{k}$ and any $x \colon \spec k \rmono S$, the fibre $C_x$ of $C \to S$ over $x$ is a \emph{smooth connected projective}\footnote{We haven't defined these formally, but they are analogous to the classical notions for varieties} curve of genus $g$ over $\spec k$.
\end{example}

\begin{fact}[Deligne-Mumford]
There is (almost) a scheme $\widetilde{M_g}$ such that there's a bijection
$$M_g\prs{S} = \hom\prs{S, M_g} \leftrightarrow M_g\prs{S}$$
which is functorial in $S$.
\end{fact}

\begin{remark}
We want to understand $\dim M_g$. We can compute it by computing $$\dim_{\CC} \hom_{C_0}\prs{\spec\prs{\quot{\CC\brs{x}}{x^2}}, M_g \times_{\ZZ} \CC}$$
the dimension of the tangent space of $M_g$ at $C_0 \in M_g\prs{\CC}$.
We have
$$\hom_{C_0}\prs{\spec\prs{\quot{\CC\brs{x}}{x^2}}, M_g \times_{\ZZ} \CC} \subseteq M_g\prs{\spec\prs{\quot{\CC\brs{x}}{x^2}}} \text{,}$$
where the latter can be computed via deformation theory from commutative algebra.
\end{remark}

% 30.4.2020

Let $f \colon X \to Y$ be a morphism of schemes and let $\Delta \colon X \to X \times_Y X$ be the ``diagonal'' corresponding to $f$ on each factor.

\begin{definition}
$f \colon X \to Y$ is \stress{separated} if $\Delta$ is a closed immersion.
\end{definition}

\begin{definition}
A scheme $X$ is \stress{separated} if $X \to \spec \ZZ$ is separated.
\end{definition}

\begin{notation}
Let $k \in \catname{Field}$, we may write $X \times_k Y$ to mean $X \times_{\spec k} Y$.
\end{notation}

\begin{example}
Let $k \in \catname{Field}$ and let $X$ be the affine line with double origin over $k$.
Then $X$ is \emph{not} separated over $k$.

$X \times_k X$ is geometrically the plane with doubled axes and four different origins.
We claim $\Delta\prs{X}$ isn't closed in $X \times_k X$. We can think of the fibre product as gluing of four planes so we get maps
$$\AA_k^2 \rmono \coprod_{i \in [4]} \AA_k^2 \to X \times_k X \text{.}$$
Now, pulling back $\Delta\prs{X}$ to $\AA_k^2$ we get the diagonal minus the origin, so $X$ isn't separated.
\end{example}

\begin{proposition}
Let $f \colon \spec A \to \spec B$. Then $f$ is separated.
\end{proposition}

\begin{proof}
Let $\phi \colon B \to A$ induce $f$. Then $\Delta^{\#} \colon A \otimes_B A \to A$ send $a \otimes a'$ to $a a'$.
This is surjective.
Hence $\Delta$ is a closed immersion, hence $f$ is separated.
\end{proof}

\begin{corollary}
If $f \colon X \to Y$ is a morphism of schemes, then $f$ is separated iff $\Delta\prs{X}$ is closed in $X \times_Y X$.
\end{corollary}

\begin{proof}
Suppose $\Delta\prs{X} \subseteq X \times_Y X$ is closed.
We want to show the following.
\begin{enumerate}
    \item $\Delta \colon X \to \Delta\prs{X}$ is a homeomorphism onto the image.
    Sending $X$ to $\Delta\prs{X}$ and then projecting to $X$ via one of the projections is a homeomorphism. This is clear. Hence $\Delta \colon X \to \Delta\prs{X}$ is an injection, and is therefore an homeomorphism onto the image.
    \item $\OOO_{X \times_Y X} \repi \Delta_*\OOO_X$ is surjective. It's enough to show this on stalks. Check this at a point $\Delta\prs{P} \in \Delta\prs{X}$. Choose $U \subseteq X$ open affine such that $P \in U$ and $f\prs{U} \subseteq V$ where $V \subseteq Y$ is open affine.
    
    Then $U \times_Y U \subseteq X \times_Y X$ is an open affine subset of $\Delta\prs{P}$.
    The corresponding map on stalks is surjective by the previous propoisition.
    Hence $\prs{\OOO_{X \times_Y X}}_{\Delta\prs{P}} \to \prs{\Delta_*, \OOO_X}_{\Delta\prs{P}}$ is surjective.
\end{enumerate}
\end{proof}

\section{Valuation Rings}

Let $k \in \mrm{Field}$ and $G$ a totally ordered abelian group (e.g. $\ZZ, \RR, \RR^n$).

\begin{definition}[Valuation]
A \stress{valuation on $k$} is a map $v \colon k^\times \to G$ such that
\begin{enumerate}
    \item $v\prs{xy} = v\prs{x} + v\prs{y}$.
    \item $v\prs{x+y} \geq \min\prs{v\prs{x}, v\prs{y}}$.
\end{enumerate}
\end{definition}

\begin{remark}
Clearly, $v\prs{1} = 0$ and $v\prs{r^{-1}} = -v\prs{r}$.
\end{remark}

\begin{definition}[Discrete Valuation]
A \stress{discrete valuation} is a valuation into $\ZZ$.
\end{definition}

\begin{example}
Let $k = \CC\prs{\prs{t}}$, we can define the valuation
$$v \prs{\sum_{i \geq i_0}^\infty a_i \cdot t^i} = i_0$$
is a discrete valuation.
\end{example}

\begin{definition}[Valuation Ring]
Let $v \colon k \to G$ be a valuation. The \stress{valuation ring of $v$} is
$$R_v \ceq \set{v \in k^\times}{v\prs{r} \geq 0} \cup \set{0} \text{.}$$
\end{definition}

\begin{remark}
$R_v$ is a local ring with maximal ideal
$$\mm_v = \set{R \in R}{v\prs{r} > 0} \cup \set{0} \text{.}$$
\end{remark}

\begin{definition}[Discrete Valuation Ring]
A ring $R$ is a \stress{discrete valuation ring} if there's a valuation $v$ such that $R \cong R_v$.
\end{definition}

\begin{lemma}
A ring $R$ is a discrete valuation ring iff $R$ is a PID with a unique non-zero maximal ideal.
\end{lemma}

\begin{proof}
Use the fact that ideals of $R$ are of the form $\mm_v^n$ where $\mm_v = \prs{t}$ and $v\prs{t} > 1$.
\end{proof}

\begin{example}
Let $k = \QQ$ and $p \in \ZZ$ prime. Define
$v \colon \QQ^\times \to \ZZ$ by $p^k \frac{m}{n} \mapsto k$ when $\gcd\prs{p,mn} = 1$. This is the \stress{$p$-adic valuation} and induces the DVR $R = \ZZ_{\prs{p}}$ of the \stress{$p$-adic integers}.
\end{example}

\begin{remark}
If $R$ is a DVR then $\spec R = \set{\prs{0}, \mm}$ with $\mm$ a maximal ideal.
\end{remark}

\begin{example}
Let $R = \CC\brs{\brs{t}} \subseteq \CC\prs{\prs{t}} \cong k$. We have an induced map $\spec k \rmono \spec R$.
\end{example}

\begin{theorem}[Valuative Criterion for Separatedness]
Let $f \colon X \to Y$ be a morphism of schemes and assume $X$ is Noetherian. Then $f$ is separated iff for all valuation rings $R \subseteq K$ (where $K = \mrm{Frac}\prs{R}$) and all commutatives diagrams
$$
\begin{tikzcd}
\spec K \arrow[r] \arrow[d, hookrightarrow] & X \arrow[d] \\ \spec R \arrow[r] & Y
\end{tikzcd}
$$
there's at most one morphism $\spec R \to X$ such that the resulting diagram commutes.
\end{theorem}

\begin{proof}
See Hartshorne.
\end{proof}

\begin{corollary}
Under Noetherian hypothesis we have the following.
\begin{enumerate}
    \item Open and closed immersions are separated.
    \item A composition of separated morphisms is separated.
    \item Any base change of a separated morphism is separated.
    I.e. if $f \colon X \to Y$ is separated and $g \colon Y' \to Y$ is a morphism then the induced map $f' \colon X \times_Y Y' \to Y'$ is separated.
    \item Products of separated maps are separated. If $f \colon X \to Y$ and $f' \colon X' \to Y'$ are separated maps of $S$-schemes then
    $$X \times_S X' \xrightarrow{f \times f'} Y \times_S Y'$$ is separated.
    \item If $X \xrightarrow{f} Y \xrightarrow{g} Z$ and $gf$ is separated then $f$ is separated.
    \item $f \colon X \to Y$ is separated iff there's an open cover (equivalently, for any open cover) $\set{Y_i}_{i \in I}$ of $Y$ such that $f^{-1}\prs{Y_i} \to Y_i$ (the restriction of $f$) is separated for every $i \in I$.
\end{enumerate}
\end{corollary}

We prove some of the parts of the corollary.

\begin{proof}
6. Assume $f$ is separated, we want to show $\rest{f}{f^{-1}\prs{Y_i}} \to Y_i$ is separated. We want to use the valuative criterion. Examine a commutative diagram
$$
\begin{tikzcd}
\spec k \arrow[r] \arrow[d, hookrightarrow] & f^{-1}\prs{Y_i} \arrow[d] \\ \spec k \arrow[r] & Y_i
\end{tikzcd}
$$
We add arrows and use the valuative criterion to say there's at most one dotted arrow
$$
\begin{tikzcd}
\spec k \arrow[r] \arrow[d, hookrightarrow] & f^{-1}\prs{Y_i} \arrow[d] \arrow[r, hookrightarrow] & X \arrow[d, "f"] \\ \spec k \arrow[dotted, rru] \arrow[r] & Y_i \arrow[r, hookrightarrow] & Y
\end{tikzcd}
$$
from which there's at most one morphism $\spec k \to f^{-1}\prs{Y_i}$ such that the following commutes.
$$
\begin{tikzcd}
\spec k \arrow[r] \arrow[d, hookrightarrow] & f^{-1}\prs{Y_i} \arrow[d] \arrow[r, hookrightarrow] & X \arrow[d, "f"] \\ \spec k \arrow[dotted, ru] \arrow[r] & Y_i \arrow[r, hookrightarrow] & Y
\end{tikzcd}
$$
Then by the valuative criterion we're done.

1. Showing that a closed immersion is separated is also done via the valuative criterion by examining the following diagram.

$$
\begin{tikzcd}
\spec K \arrow[d, hookrightarrow] \arrow[rr, bend left = 30] & X_0 \arrow[dr, hookrightarrow] \arrow[r, hookrightarrow] & X \arrow[d, hookrightarrow] \\ \spec R \arrow[rr] & & Y
\end{tikzcd}
$$
\end{proof}

\begin{definition}
A morphism $f \colon X \to Y$ is \stress{proper} if it satisfies the following.
\begin{enumerate}
    \item $f$ is separated.
    \item $f$ is of finite type.
    \item $f$ is \stress{universally closed} in the sense that for every $Y' \to Y$, the base change $f'$ is a closed map on the topological spaces.
\end{enumerate}
\end{definition}

\begin{example}
Let $k \in \catname{Field}$ and $X = \AA_k^1$. Then $X$ is not proper over $k$.
We have $\AA_k^1 \times_k \AA_k^1 = \AA_k^2$. Writing $f \colon \AA_k^1 \to \spec k$ we get $f' \colon \prs{x,y} \mapsto y$. Let $Z = \spec\prs{\quot{k\brs{x,y}}{xy-1}}$.
Then $f'\prs{Z} \cong \AA_k^1 \setminus \set{0}$ is not closed, so $\AA_k^1 \to k$ is not proper.
\end{example}

\begin{definition}[Specialisation]
Let $X_0, x_1 \in X$. If $x_0 \in \overline{\set{x_1}}$ then $x_1$ \stress{specialises to $x_0$}.
\end{definition}

\begin{definition}[Dominating Ring]
Say $A,B \subseteq K$ are local rings in a field. We say $B$ \stress{dominates $A$} if $A \subseteq B$ and $\mm_A = \mm_B \cap A$.
\end{definition}

\begin{theorem}[A-M Section 4]
Let $K$ be a field. A local ring $R \subseteq K$ is a valuation ring of $K$ iff $R$ is maximal with respect to domination.
\end{theorem}

\begin{corollary}
Every local ring in $K$ is contained in some valuation ring.
\end{corollary}

\begin{lemma} \label{lemma:valuative_for_properness}
Let $f \colon X \to Y$ quasi-compact. Then $f\prs{X}$ is closed in $Y$ iff $f\prs{X}$ is stable under specialisation.
\end{lemma}

\begin{proof}
See Hartshorne.
\end{proof}

\begin{theorem}[Valuative Criterion for Properness]
Let $f \colon X \to Y$ be of finite type and assume $X$ is Noetherian. Then $f$ is proper iff for all valuation rings $R \subseteq K$ and all commuting diagrams
$$
\begin{tikzcd}
\spec K \arrow[r] \arrow[d, hookrightarrow] & X \arrow[d] \\ \spec R \arrow[r] & Y
\end{tikzcd}
$$
there's a unique $\psi \colon \spec R \to X$ making the diagram
$$
\begin{tikzcd}
\spec K \arrow[r] \arrow[d, hookrightarrow] & X \arrow[d] \\ \spec R \arrow[r] \arrow[ur, dotted] & Y
\end{tikzcd}
$$
commute.
\end{theorem}

\begin{proof}
\begin{itemize}
\item
Let $U \ceq \spec K$ and $T \ceq \spec R$.

If $f$ is proper, it's separated so by the valuative criterion for separatedness it suffices to show that for every valuation ring $R$ with $K \ceq \Frac\prs{R}$ and solid commutative diagram as follows there exists a dotted morphism $\psi$ such that the diagram commutes.
\[
\begin{tikzcd}
U \arrow[r] \arrow[d] & X \arrow[d,"t"] \\ T \arrow[r] \arrow[ur, dotted,"\psi"] & Y
\end{tikzcd}
\]
Let $X_T \ceq X \times_Y T$, let $f' \colon X_T \to T$ correspond to $f$ and let $t_1 \in U$. We have a morphism $\phi \colon U \to X_T$ by the universal property. Let $x_1 = \phi\prs{t_1}$. We have $Z \ceq \overline{\set{x_1}} \stackrel{\text{closed}}{\rmono} X_T$ and that $Z$ is integral. $f$ is proper so $f'$ is closed. Hence $f'\prs{Z}$ is closed in $T$ and contains the generic point, hence $f'\prs{Z} = T$.
Hence there's $x_0 \in X_T$ such that $f'\prs{x_0} = t_0$ where $t_0$ is the unique closed point of $T$ (corresponding to the maximal ideal).

$Z$ is integral with function field $k\prs{x_1} \rmono K$. We get that the morphism
\begin{align*}
Z &\to T \\
R &\xrightarrow{\substack{\text{local} \\ \text{homo.}}} \OOO_{Z, x_0}
\end{align*}
can be post-composed to get a morphism to $k\prs{x_1}$ and then to $K$.
Hence $\OOO_{Z,x_0}$ dominated $R$.
Because $R$ is a valuation ring, it's maximal with respect to domination, hence the map $R \to \OOO_{Z,x_0}$ is an isomorphism.

Define
\begin{align*}
T &\to Z \\
t_i &\mapsto x_i \\
\OOO_{Z,x_0} &\to X \text{.}
\end{align*}
We get a map $\spec R = T \to \spec\OOO_{Z,x_0} \to Z \text{.}$
We post compose to get \[\psi \colon T \to Z \to \rmono X_T \to X \text{.}\]
\item Conversely, suppose there's a unique $\psi$ in all cases. We need to show $f$ is universally closed.
Let $Y' \in \Sch\prs{Y}$ and view the following commutative diagram.
\[
\begin{tikzcd}
X' \arrow[r, "f'"] \arrow[d] & Y' \arrow[d] \\
X \arrow[r, "f"] & Y
\end{tikzcd}
\]
Let $Z \subseteq X'$ closed. $f$ is of finite type hence by an exercise in Hartshorne II.3, $f'$ is of finite type. Then $\rest{f'}{Z}$ is of finite type (since closed immersions are of finite type and compositions of finite type morphisms are of finite type).
Hence $\rest{f'}{Z}$ is quasi-compact.

By \ref{lemma:valuative_for_properness}, $f'\prs{Z}$ is then closed iff it's stable under specialisation.
Let $z_1 \in Z$ and $y_1 = f'\prs{Z}$. Suppose $y_1 \rightsquigarrow y_0$ ($y_1$ specialises to $y_1$).
Let $\OOO$ be the local ring of $y_0$ in $\overline{\set{y_1}}$.
Hence \[\Frac\prs{\OOO} = k\prs{y_1} \subseteq k\prs{z_1} \coloneqq K \text{.}\]
Let $R$ be a valuation ring in $K$ dominating $\OOO$.
Let $U \ceq \spec K$ and $T \ceq \spec R$. We get the following commutative diagram.
\[
\begin{tikzcd}
U \arrow[r] \arrow[d, hookrightarrow] & Z \arrow[d] \arrow[r, hookrightarrow] & X' \arrow[dl, "f'"] \arrow[r] & X \arrow[d, "f"] \\
T \arrow[r] & Y' \arrow[rr] & & Y
\end{tikzcd}
\]
We get an induced map $T \to X$ and we want to show this factors through $Z$. This is true because $Z$ is closed and because closed points of $T$ are in the closure of the generic point which is in $K$.

Set $z_0 = \psi\prs{t_0}$. We get $f'\prs{z_0} = y_0$. Hence $f\prs{Z}$ is stable under specialisation. By \ref{lemma:valuative_for_properness} then $f'\prs{Z}$ is closed so $f'$ is closed so $f$ is universally closed.

Hence $f$ is proper.
\end{itemize}
\end{proof}

\begin{corollary}
Assume $f \colon X \to Y$ under suitable Noetherian hypothesis on $X,Y$.
\begin{enumerate}
    \item Closed immersions are proper.
    \item Properness is stable under base-change.
    \item Fibred products of proper morphisms are proper.
    \item If $X \xrightarrow{f} Y \xrightarrow{g} Z$, $gf$ is proper and $g$ is separated, then $f$ is also proper.
    \item Properness is local on the base, as before.
\end{enumerate}
\end{corollary}

\begin{definition}[Projective Space of Affine Schemes]
Let $A \in \catname{Ring}$, we define \stress{the $n$\textsuperscript{th} projective space over $A$} to be
$$\PP_A^n = \mrm{Proj}\prs{A \brs{x_0, \ldots, x_n}} \text{.}$$
\end{definition}

\begin{remark}
A map $A \to B$ induces
$$\PP_{B}^n \xrightarrow{\sim} \PP_A^n \times_{\spec A} \spec B \text{.}$$
\end{remark}

\begin{definition}[Projective Space of Schemes]
Let $X$ be a scheme, we define $\PP_{X}^{n} = \PP_{\ZZ}^n \times_{\spec \ZZ} X$.
\end{definition}

\begin{definition}[Projective Morphism]
A morphism $f \colon X \to Y$ is \stress{projective} if it facotrs as
$X \stackrel{i}{\rmono} \PP_Y^n \to Y$ for some $n \in \NN$ and where $i$ is a closed immersion.
\end{definition}

\begin{definition}[Quasi-projective Morphism]
$f \colon X \to Y$ is \stress{quasi-projective} if it factors as $X \stackrel{i}{\rmono} X' \xrightarrow{f'} Y$ with $i$ an open immersion and $f'$ projective.
\end{definition}

\begin{claim}
Let $A \in \catname{Ring}$ and $S = \bigoplus_{d \in \NN} S_d$ an $\NN$-graded ring with $S_0 = A$. Assume $S$ is finitely-generated as an $A$-algebra by elements of $S_1$.
Then that the natural map
$\mrm{Proj}\prs{S} \to \spec A$ is projective.
\end{claim}

\begin{proof}
Let $S' \ceq A\brs{x_0, \ldots, x_n}$, there's a surjective map $S' \repi S$ by taking degree $1$ terms to degree $1$ terms which include the generating set in $S_1$.
Then the map factors as
$$\mrm{Proj} S \rmono \mrm{Proj} S' = \PP_A^n \to \spec A \text{.}$$
\end{proof}

\begin{theorem}
A Projective morphism of Noetherian schemes is proper.
\end{theorem}

\begin{proof}
Let $f \colon X \to Y$ a projection morphism of Noetherian schemes. We can factor $f \colon X \rmono \PP_Y^n \to Y$ where $\PP_Y^n = \PP_\ZZ^n \times_{\spec \ZZ} Y$.
It's neough to show $\PP_\ZZ^n \to \spec \ZZ$ is proper, which then implies that the base change is proper. Because $X \rmono \PP_Y^n$ is proper this gives the result.

Write \[\PP_\ZZ^n = \bigcup_{i = 0}^n D^+\prs{x_i} = \bigcup_{i = 0}^n V_i\]
where $V_i \cong \spec \ZZ\brs{\frac{x_0}{x_i}, \ldots, \frac{x_n}{x_i}}$.
Hence $\PP_\ZZ^n \to \spec \ZZ$ is of finite type.

We have to show that for every diagram
\[
\begin{tikzcd}
\spec K = U \arrow[d, hookrightarrow] \arrow[r] & X = \PP_\ZZ^n \arrow[d] \\
\spec R = T \arrow[r] \arrow[ur, dotted, "\psi"] & \spec \ZZ
\end{tikzcd}
\]
of solid arrows there's a unique $\psi$, where $R \subseteq K$ is a valuation ring.

Let $u_1 \in U$ and $p_1$ its image. We may assume by induction $p_1 \notin X \setminus V_i \cong \PP_\ZZ^{n-1}$. Then $\frac{x_i}{x_j}$ are invertible in $\OOO_{X, p_1}$.

We have $k\prs{p_1} \subseteq K$. Write $f_{i,j}$ to be the image of $\frac{x_i}{x_j}$ in $K$. Let $v \colon K^\times \to G$ be a valuations and let $g_i \ceq v_i\prs{f_{i,0}}$ for $i \in \set{0, \ldots, n}$.
Let $g_r$ be minimal.
Then $v\prs{f_{i,r}} = g_i - g_r$ because $f_{i,m} = f_{i,j} f_{j,m}$.
Then $v\prs{f_{i,r}} \geq 0$. So, define
\begin{align*}
\phi \colon \ZZ\brs{\frac{x_0}{x_r}, \ldots, \frac{x_n}{x_r}} &\to R\\
\frac{x_i}{x_r} &\mapsto f_{i,r} \text{.}
\end{align*}
This gives a map $T \to V_r \rmono \PP_\ZZ^n$ which we call $\psi$.
\end{proof}

\begin{definition}[Variety]
A \stress{variety} is an integral separated scheme $X$ of finite type over a field $\spec k$ where $k = \bar{k}$.
\end{definition}

\begin{remark}
Usually, one omits the requirement $k = \bar{k}$ but requires that $X$ is \emph{geometrically integral} which we don't define now.
\end{remark}

\begin{proposition}
Let $k = \bar{k}$. Recall the functor $t$ from classical algebraic varieties to $\Sch_k$.

The image of $t$ is the set of quasi-projective integral schemes over $k$ (in our new terminology, quasi-projective varieties over $k$).

Moreover, if $V$ is projective then so is $t\prs{V} \to \spec k$ is projective.
\end{proposition}

\chapter{Sheaves of Modules}

Let $\prs{X, \OOO_X}$ be a ringed space.

\begin{definition}[$\OOO_X$-Module]
An $\OOO_X$-module is a sheaf $\FFF$ on $X$ of abelian groups such that for every open sets $U \subseteq X$ and $V \subseteq X$:
\begin{enumerate}
\item $\FFF\prs{U}$ is an $\OOO_X\prs{U}$-module in the sense that we're given a map
\[\OOO_X\prs{U} \times \FFF\prs{U} \to \FFF\prs{U}\]
for each $U$.
\item The morphism $\FFF\prs{U} \to \FFF\prs{V}$ is a morphism of $\OOO_X\prs{U}$-modules.
\end{enumerate}
\end{definition}

\begin{definition}[Morphism of $\OOO_X$-Modules]
A \stress{morphism $f \colon \FFF \to \GGG$ of $\OOO_X$-modules} is a morphism of sheaves such that each $f_U \colon \FFF\prs{U} \to \GGG\prs{U}$ is $\OOO_X\prs{U}$-linear.
\end{definition}

\begin{lemma}
If $F$ is a presheaf of abelian groups satisfying conditions 1,2 above then $F^+$ naturally carries a structure of $\OOO_X$-modules.
\end{lemma}

\begin{proof}
We need to define a dotted map in the following diagram
\[
\begin{tikzcd}
\OOO_X \times \FFF \arrow[r] \arrow[d] & \FFF \arrow[d] \\
\OOO_X \times \FFF^+ \arrow[r,dotted] & \FFF^+
\end{tikzcd}
\]
where the sheaves are considered as sheaves of sets.
This is given by the fact that sheafification commutes with direct products and by the universal property of sheafification.
One can check that this map satisfies the desired axioms.
\end{proof}

\begin{fact}
\begin{enumerate}
\item If $f \colon \FFF \to \GGG$ is a morphism of $\OOO_X$-modules then so are $\ker\prs{f}, \coker\prs{f}, \im\prs{f}$.
\item %TODO continue
\end{enumerate}
\end{fact}

%7.5.20202

%TODO fill in beginning of lecture

\begin{definition}[Invertible Sheaf]
A locally free sheaf of rank $1$ is called \stress{an invertible sheaf (alt. ``line bundle'')}.
\end{definition}

\begin{remark}
The reason that it's called an invertible sheaf is that $\FFF \tensor \abhom\prs{\FFF,\OOO_X} \cong \OOO_X$.
\end{remark}

\begin{definition}[$\OOO_X$-Submodule]
An $\OOO_X$-submodule is an $\OOO_X$-module $\FFF$ such that every $\FFF\prs{U}$ is an $\OOO_X\prs{U}$-submodule with the given action.

This is also called a \stress{sheaf of ideals on $X$}.
\end{definition}

\begin{proposition}
Let $f \colon \prs{X, \OOO_X} \to \prs{Y, \OOO_Y}$ be a morphism of ringed spaces and let $\FFF \in \OOO_X\Mod$ (the category of $\OOO_X$-modules).

Then $f_*\FFF$ is an $f_* \OOO_X$-module.
\end{proposition}

\begin{proof}
We have $\prs{f_* \FFF}\prs{U} = \FFF\prs{f^{-1}\prs{U}}$ which is a module over $\OOO_X\prs{f^{-1}\prs{U}} = \prs{f_*\OOO_X}\prs{U}$.
But, $f_* \OOO_X$ is an $\OOO_Y$-module via
\[f^{\#} \colon \OOO_Y \to f_* \OOO_X \text{.}\]
So, $f_* \FFF$ is an $\OOO_Y$-module in a natural way.
\end{proof}

\begin{proposition}
Let $f \colon \prs{X, \OOO_X} \to \prs{Y, \OOO_Y}$ be a morphism of ringed spaces and let $\GGG \in \OOO_Y\Mod$.

Then $f^{-1}\GGG$ is an $f^{-1} \OOO_Y$-module.
\end{proposition}

\begin{proof}
$\prs{f^{-1}\GGG}\prs{U} \cong \lim_{V \supseteq f\prs{U}} G\prs{V}$ is a module over $\lim_{V \supseteq f\prs{U} \OOO_Y\prs{V}} \cong \prs{f^{-1}\OOO_Y}\prs{U}$.

Adjunction applied to $f^\#$ gives $f^{-1}\OOO_Y \to \OOO_X$.
Set $f^* \GGG \ceq f^{-1}G \tensor_{f^{-1}\OOO_Y} \OOO_X$.
\end{proof}


\begin{remark}
If $f$ is an open immersion of schemes $f \colon U \rmono X$ then $f^* \GGG \cong \rest{\GGG}{U}$.
\end{remark}

\begin{fact}
There's a natural isomorphism of groups
\[\abhom_{\OOO_X}\prs{f^* \GGG, \FFF} \cong \abhom_{\OOO_Y}\prs{G, f_*\FFF} \text{.}\]
\end{fact}

\begin{definition}
Let $A \in \Ring$ and $M \in \A\Mod$. Let $\tilde{M}$ be the sheaf on $\spec A$:
\[\tilde{M}\prs{U} = \set{s \colon U \to \coprod_{\pp \in \spec A} M_\pp}{\substack{s\prs{\pp} \in M_\pp \\ \star}}\]
where $\star$ is the property that for all $\pp \in U$ there's a neighbourhood $V \subseteq U$ of $\pp$ and there are $m \in M, f \in A$ such that for all $\qq \in V$ it holds that $f \notin \qq$ and $s\prs{\qq} = \frac{m}{f} \in M_\qq$.
\end{definition}

\begin{proposition}
Let $X \ceq \spec A$.
\begin{enumerate}
\item $\tilde{M}$ is an $\OOO_X$-module.
\item $\prs{\tilde{M}}_\pp \cong M_\pp$ as $A_\pp$-modules for every $\pp \in X$.
\item $\tilde{M}\prs{D\prs{f}} \cong M_f$ as $A_f$-modules, for every $f \in A$, where $D\prs{f}$ is the basic open in $X$.
\item $\tilde{M}\prs{X} \cong M$.
\end{enumerate}
\end{proposition}

\begin{remark}
\begin{align*}
A\Mod &\to \OOO_{\spec A}\Mod \\
M &\mapsto \tilde{M}
\end{align*}
is a functor.
\end{remark}

\begin{proposition}
\begin{enumerate}
\item The functor $M \to \tilde{M}$ is exact and fully-faithful.
\item $\prs{M \tensor_A N}^\sim \cong \tilde{M} \tensor_{\OOO_X} \tilde{N}$.
\item $\prs{\bigoplus_{i \in I} M_i}^{\sim} \cong \bigoplus_{i \in I} \tilde{M}_i$.
\item Let $\phi \colon A \to B$ a ring morphism and let $N \in B\Mod$. Let $f \colon \spec B \to \spec A$ the induced map on spectra.
Then $f_* \tilde{N} \cong \widetilde{\prs{_{A}N}}$ where ${}_A N$ is $N$ thought of as an $A$-module.
\item If $M \in A\Mod$ then $f^*\tilde{M} \cong \widetilde{\prs{M \tensor_A B}}$.
\end{enumerate}
\end{proposition}

\begin{proof}
\begin{enumerate}
\item
\begin{description}
\item[Fully-faithfulness:]
We want to show $\hom_A\prs{M,N} \to \hom_{\OOO_X}\prs{\tilde{M},\tilde{N}}$ is bijective. The inverse map is taking global sections.
\item[Exactness:]
Let \[0 \to M' \to M \to M'' \to 0\]
be exact. We want to show that the induced sequence
\[0 \to \tilde{M}' \to \tilde{M} \to \tilde{M}'' \to 0\]
is exact.
We can check this on stalks which gives the result.
\end{description}
\item We have a natural map
\[\prs{M \tensor_A N}^{\sim} \to \tilde{M} \tensor_{\OOO_X} \tilde{N}\]
by extending linearly.
We have to show this is an isomorphism on the stalks. Because localisation commutes with tensor products we get
\[\prs{M \tensor_A N}_{\pp} \cong M_\pp \tensor_{A_\pp} N_\pp\]
so the natural map is an isomorphism on stalks, and is therefore an isomorphism.

\item
Exercise.
\item
Exercise.
\item
Exercise.
\end{enumerate}
\end{proof}

\begin{definition}[Quasi-Coherent $\OOO_X$-Module]
Let $X \in \Sch$ and $\FFF \in \OOO_X\Mod$. Then $\FFF$ is called \stress{quasi-coherent} if there's an open affine cover $\prs{U_i}_{i \in I}$ with $U_i \cong \spec A_i$ of $X$ such that $\rest{\FFF}{U_i} \cong \tilde{M}_i$ for some $A_i$-module $M_i$, for all $i \in I$.
\end{definition}

\begin{definition}[Coherent $\OOO_X$-Module]
A quasi-coherent $\OOO_X$-module $\FFF$ is \stress{coherent} if the $M_i$ cacn be taken to be finite $A_i$-modules.
\end{definition}

\begin{remark}
$\OOO_X$ is coherent for every $X \in \Sch$.
\end{remark}

\begin{example}
Let $X = \spec A$ and $Y \subseteq X$ a closed subscheme defined by $I \ideal A$. Then
\[i_* \OOO_Y \cong \widetilde{\prs{\quot{A}{I}}}\]
where we think of $\quot{A}{I}$ as an $A$-module.

This is coherent.
\end{example}

\begin{definition}[Quasi-Coherent Sheaf]
A quasi-coherent sheaf on $X$ is a quasi-coherent $\OOO_X$-module.
\end{definition}

\begin{notation}
Denote by $\catname{Qcoh}\prs{X}$ the category of quasi-coherent sheaves on $X$.
\end{notation}

\begin{lemma}\label{lemma:quasi-coherent}
Let $X = \spec A$, $f \in A$, $D\prs{f} \subseteq X$ and $\FFF \in \catname{Qcoh}\prs{X}$.
\begin{enumerate}
\item If $s \in \Gamma\prs{X,\FFF}$ is such that $\rest{s}{D\prs{f}} = 0$ then $f^n s = 0$ for some $n > 0$.
\item If $t \in \FFF\prs{D\prs{f}}$ then for some $n>0$, $f^n t$ extends to a global section in $\Gamma\prs{X, \FFF}$.
\end{enumerate}
\end{lemma}

\begin{proof}
Cover $X$ with over affine $\prs{V_i}_{i \in I}$ such that $V_i \cong \spec B_i$ and $\rest{\FFF}{V_i} \cong \tilde{M}_i$ for $M$ a $B_i$-module.

For every $g \in A$ such that $D\prs{g} \subseteq V_i$, and call $\iota \colon D\prs{g} \rmono V_i$ the inclusion. We get a morphism $B_i \to A_g$.
So \[\rest{\FFF}{D\prs{g}} = \rest{\prs{\rest{\FFF}{V_i}}}{{D\prs{g}}} = \rest{\tilde{M_i}}{D\prs{g}} = \iota^*\tilde{M} = \prs{M \tensor_B A_g} \text{.}\]
Hence there's an open affine cover $\prs{D\prs{g_i}}_{i \in I'}$ of $X$ which is and such that $\rest{\FFF}{D\prs{g_i}} \cong \tilde{M}_i$ and $M_i$ is a $A_{g_i}$-module. We can take $I'$ to be finite because $X$ is affine and therefore quasi-compact.

\begin{enumerate}
\item If $s \in \Gamma\prs{X,\FFF}$ with $\rest{s}{D\prs{f}} = 0$, let $s_i \ceq \rest{s}{D\prs{g_i}} \in \tilde{M}_i \prs{D\prs{g_i}} \cong M_i$.
Then
\[D\prs{f} \cap D\prs{g_i} = D\prs{f g_i}\]
so
\[\rest{\FFF}{D\prs{f g_i}} = \rest{\tilde{M}_i}{D\prs{f g_i}} \cong \prs{\tilde{M}_i}_f \text{.}\]
Hence $s_i = 0 \in \prs{M_i}_f$. Hence $f^n s_i = 0$ for some $n \in \NN_+$. Because $I'$ is finite we get the result by taking the maximal power.

\item Given $t \in \FFF\prs{D\prs{f}}$ let $\tilde{t}_i \in \FFF\prs{D\prs{f g_i}} \cong \prs{M_i}_f$ the restriction of $t$.
Then there are $t_i \in M_i$ such that $t_i = f^n \tilde{t}_i$. Then the $t_i$ agree on the double intersections $D\prs{f g_i g_j}$.
Then $f^m \prs{t_i - t_j} = 0$ on $D\prs{g_i g_j}$ by part 1.
Hence $f^m t_i$ glue to a section $s \in \FFF\prs{X}$ such that \[\rest{s}{D\prs{f}} = f^m \set{t_i} = f^{m+n} t\]
where $\set{t_i}$ is the gluing of the $t_i$'s.
\end{enumerate}
\end{proof}

%13.5.2020

\begin{proposition}
Let $X \in \Sch$ and let $\FFF \in \OOO_X\Mod$. Then:
\begin{enumerate}
\item $\FFF$ is quasi-coherent iff for all $U \subseteq X$ open affine there exists $M \in A\Mod$ such that $\rest{\FFF}{U} \cong \tilde{M}$.
\item Assume $X$ is Noetherian. $\FFF$ is coherent iff for all $U \subseteq X$ open affine there exists $M \in A\Mod^{\mrm{fg}}$ such that $\rest{\FFF}{U} \cong \tilde{M}$.
\end{enumerate}
\end{proposition}

\begin{proof}
\begin{enumerate}
\item Assume $\FFF$ is quasi-coherent and let $U \subseteq X$ open and isomorphic to $\spec A$. Then $\rest{\FFF}{U}$ is quasi-coherent because we can intersect a covering of $X$ with $U$ to get an appropriate cover. We can therefore assume by replacing $X$ with $U$ that $X$ is affine with $\spec A$.

Let $M = \Gamma\prs{X, \FFF}$, we want to show $\FFF \cong \tilde{M}$.
We have $\alpha \colon \tilde{M} \to \FFF$ from the universal property of the sheafification.
We have that the sections of $\tilde{M}$ at $U$ are $\set{s \colon U \to \coprod_p M_p}{s\prs{p} = \frac{m}{f} \text{ locally}}$.
We can then send $s$ to $\frac{m}{f}$ in $\FFF\prs{U}$.

Cover $X$ with $D\prs{g_i}$ such that $\rest{\FFF}{D\prs{g_i}} \cong \tilde{M_i}$ where $M_i \in A_{g_i}\Mod$.
By \ref{lemma:quasi-coherent} we get $\FFF\prs{D\prs{g_i}} \cong M_{g_i}$. Then $M_{g_i} \cong M_i$ because we can recover both as global sections of the same sheaf.

Hence $\alpha$ is an isomorphism over $D\prs{g_i}$ for all $i$, hence it's an isomorphism on every stalk and hence an isomorphism.

\item Assume $\FFF$ is coherent and reduce similarly to the case where $\FFF$ is affine. It's then also quasi-coherent so by the above $\FFF \cong \tilde{M}$ where $M \in A\Mod$. There are $\prs{g_i}_{i \in [n]}$ which generate $A$ such that $M_{g_i}$ are finitely-generated $A_{g_i}$-modules.
Then $M_{g_i}$ are Noetherian hence so is $M$ (via the same proof as for rings).
\end{enumerate}
\end{proof}

\begin{proposition}
Let $X \in \Sch$ affine and let
\[0 \to \FFF' \to \FFF \to \FFF'' \to 0\]
be an exact sequence of $\OOO_X$-modules. Assume $\FFF' \in \catname{Qcoh}\prs{X}$. Then
\[0 \to \Gamma\prs{X,\FFF'} \to \Gamma\prs{X,\FFF} \to \Gamma\prs{X,\FFF''} \to 0\]
is exact.
\end{proposition}

\begin{proof}
By an exercise from Hartshorne, $\Gamma\prs{X,-}$ is left exact.
Hence it's enough to show that $\tilde{\beta} \colon \Gamma\prs{X,\FFF} \to \Gamma\prs{X,\FFF''}$ is surjective.

Let $s \in \Gamma\prs{X,\FFF''}$.
Let $x \in X$, there's $D\prs{f}$ containing $x$ such that $\rest{s}{D\prs{f}} = \beta\prs{t_x}$ for some $t_x \in \FFF\prs{D\prs{g_i}}$.

\begin{claim}
$f^n s$ lifts to $\FFF\prs{X}$ for some $n > 0$.
\end{claim}

\begin{proof}
Cover $X$ with $D\prs{g_i}$ such that $\rest{s}{D\prs{g_i}} = \beta\prs{t_i}$ with $t_i \in \FFF\prs{D\prs{g_i}}$.

On $D\prs{f} \cap D\prs{g_i} = D\prs{f g_i}$ we have that $t_x, t_i$ both map under $\beta$ to $s$. Hence $t_x - t_i \in \ker \beta \prs{D \prs{f g_i}}$ so $t_x - t_i \in \FFF'\prs{D\prs{f _i}}$.
By \ref{lemma:quasi-coherent} there's $n \in \NN$ such that $f^n \prs{t_x - t_i}$ extends to $u_i \in \FFF'\prs{D\prs{g_i}}$. There are finitely many indices $i$ so choose $n$ big enough for all of them.

Let $t_i' \ceq f^n t_i + u_i$. Then $\beta\prs{t_i'} = \beta\prs{f^n t_x} = f^n s$ on $D\prs{g_i}$.
Also $t_i' = f^n t_x$ on $D\prs{f g_i}$.

So on $D\prs{g_i g_j}$ we have that $t_i', t_j'$ both lift to $f^n s$. Then $t_i' - t_j' \in \FFF'\prs{D\prs{g_i g_j}}$ so the different is $0$ on $D\prs{f g_i g_j}$.
By \ref{lemma:quasi-coherent}, $f^m \prs{t_i' - t_j'} = 0$ for $m$ large enough, so $\set{f^m t_i'}_{i \in I}$ glue to $t'' \in \FFF\prs{X}$ such that $\beta\prs{t''} = f^{m+n}s \in \FFF''\prs{X}$.
\end{proof}

Now, cover $X$ with $D\prs{f_i}$. Then $\rest{s}{D\prs{f_i}}$ lift  to $\FFF\prs{D\prs{f_i}}$ and by the claim to $t_i \in \FFF\prs{X}$ such that $\beta\prs{t_i} = f_i^n s$.
Now
$\prs{f_1, \ldots, f_k} = A = \prs{f_i^n, \ldots, f_k^n}$ so $1 = \sum_{i} a_i f_i^n$ for some $a_i \in A$. Let $t = \sum a_i t_i \in \FFF\prs{X}$.
Then
\[\beta\prs{t} = \sum a_i \beta\prs{t_i} = \sum a_i f_i^n s = s \sum a_i f_i^n = s \text{.}\]
\end{proof}

\begin{proposition}
Let $X \in \Sch$.
\begin{enumerate}
\item Kernels, cokernels and images exist in $\catname{Qcoh}\prs{X}$.
\item Any extension of quasi-coherent sheaves is quasi-coherent. I.e. if the outer sheaves in a short exact sequence are quasi-coherent, so is the middle one.
\item If $X$ is Noetherian, 1,2 hold in $\catname{Coh}$.
\end{enumerate}
\end{proposition}

\begin{proof}
It's enough to assume $X$ is affine.
\begin{enumerate}
\item %1
Take the following exact sequence $0 \to \FFF' \to \FFF \to \FFF''$ where $\FFF,\FFF''$ are quasi-coherent. Write $\FFF \cong \tilde{M}$ and $\FFF \cong \tilde{M''}$.
The morphism $M \to M''$ has a kernel $M'$ so that $0 \to M' \to M \to M''$ is exact. By exactness of $\tilde{\phantom{F}}$ we get that $0 \to \tilde{M'} \to \tilde{M} \to \tilde{M''}$ is exact, so by fully-faithfulness $\FFF' \cong \tilde{M'}$.
\item %2
Suppose we have a short exact sequence
$0 \to \FFF' \to \FFF \to \FFF'' \to 0$ where $\FFF' \cong \tilde{M'}$ and $\FFF'' \cong \tilde{M''}$.
Taking global sections we get
\[0 \to M' \to M \to M'' \to 0\]
which is exact by the above proposition.
Then
\[0 \to \FFF' \to \tilde{M} \to \FFF'' \to 0\]
is exact. We have a functorial morphism $\alpha \colon \tilde{M} \to \FFF$. $\alpha$ on $\FFF', \FFF''$ is an isomorphism, so we have a diagram
\[
\begin{tikzcd}
0 \arrow[r] & \FFF' \arrow[r] \arrow[d] & \tilde{M} \arrow[d, "\alpha"] \arrow[r] & \FFF' \arrow[r] \arrow[d] & 0 \\
0 \arrow[r] & \FFF' \arrow[r]& \FFF \arrow[r] & \FFF'' \arrow[r] & 0
\end{tikzcd}
\]
where the vertical morphisms on the sides are isomorphisms. Then by the 5-lemma $\alpha$ is an isomorphism. Hence $\FFF$ is quasi-coherent.
\end{enumerate}
\end{proof}

\begin{proposition}
Let $f \colon X \to Y$ in $\Sch$.
\begin{enumerate}
\item If $\GGG \in \catname{Qcoh}\prs{Y}$ then $f^* \GGG \in \catname{Qcoh}\prs{X}$.

\item If $X,Y$ are Noetherian and $\GGG \in \catname{Coh}\prs{Y}$ then $f^* \GGG \in \catname{Coh}\prs{X}$.

\item If either $X$ is Noetherian or $f$ is quasi-compact and separated then if $\FFF \in \catname{Qcoh}\prs{X}$ then $f_* \FFF \in \catname{Qcoh}\prs{Y}$.
\end{enumerate}
\end{proposition}

\begin{remark}
Part 3 of the proposition fails badly for $\catname{Coh}\prs{X}$. For example, take $\spec k\brs{x} \to \spec k$.
\end{remark}

\begin{proof}
For 1,2 we reduce to the case where $X,Y$ are affine:
Take $U \subseteq Y$ open affine and examine $f^{-1}\prs{U}$. This can be covered by affines. Now $\rest{f^* \GGG}{V} = \prs{\rest{f}{V}}^* \rest{\GGG}{U}$.

\begin{enumerate}
\item We have $\GGG \cong \tilde{M}$ and $f^* \GGG \cong \prs{M \tensor_A B}^{\sim} \in \catname{Qcoh}\prs{X}$.
\item Similarly, where $M$ is finitely-generated over $A$ and hence over $B$ by Noetherianity of both.
\item We have $\rest{\prs{f_* \FFF}}{U} \cong f_* \prs{\rest{\FFF}{f^{-1}\prs{U}}}$ for $U \subseteq Y$. Cover $X$ with open affines $\prs{U_i}_{i \in [n]}$.

\begin{itemize}
\item If $f$ is separated, $U_i \cap U_j$ is affine (exercise 4.3).

\item If $X$ is Noetherian, we can cover $U_i \cap U_j$ with finitely many open affines $U_{i,j,k}$.
\end{itemize}

By the sheaf property we get the following exact sequence, where the last map is the identity if $i$ appears in the first coordinate or minus the identity if it's on the second one.

\[0 \to f_* \FFF \to \bigoplus_{i \in [n]} f_* \prs{\rest{\FFF}{U_i}} \to \bigoplus_{i,j,k} f_* \prs{\rest{\FFF}{U_{i,j,k}}}\]
The direct sums are quasi-coherent because $U_i$ are affine, so $f_* \FFF$ is quasi-coherent as the kernel of the morphism of quasi-coherent schemes.
\end{enumerate}
\end{proof}

\subsection{Ideal Sheaves}

Let $i \colon Y \rmono X$ be a closed subscheme. The \stress{ideal sheaf} $\III_Y$ of $Y$ is $\ker\prs{\OOO_X \repi i_* \OOO_Y}$. We have an exact sequence
\[0 \to \III_Y \to \OOO_X \to i_*\OOO_Y \to 0 \text{.}\]

\begin{proposition}
\begin{enumerate}
\item $\III_Y$ is quasi-coherent.
\item Any quasi-coherent sheaf of ideals on $X$ arises from some $Y$ in the above sense.
\item If $X$ is Noetherian, $\III_Y$ is coherent.
\end{enumerate}
\end{proposition}

\begin{proof}
\begin{enumerate}
\item
We know $\OOO_X$ is quasi-coherent, hence by the exact sequence we may prove $i_* \OOO_Y$ is quasi-coherent. Because $i$ is a closed immersion, it's quasi-compact and separated (e.g. in the Noetherian case, this follows from the valuative criterion).
Then by the previous proposition $i_* \OOO_Y$ is quasi-coherent from which we get the result as stated.
\item Let $\III$ be a quasi-coherent sheaf of ideals of $\OOO_X$. Take \[Y = \supp\prs{\quot{\OOO_X}{\III}} = \set{y \in X}{\prs{\quot{\OOO_X}{\III}}_y \neq 0} \text{.}\]
Locally, $\III_{U_i}$ is quasi-coherent sheaf of ideals on $\spec A_i \cong U_i$. So it's of the form $\tilde{\aa_i}$ for some $\aa_i \ideal A$.
Hence $\prs{V\prs{\aa_i}, \quot{A_i}{\aa_i}}$ is a closed subscheme of $U_i$ and these glue to give $\prs{Y, \quot{\OOO_Y}{\III}} \rmono X$.
\end{enumerate}
\end{proof}

\begin{corollary}
If $X =\spec A$ then there's a bijection between ideals $I \ideal A$ and closed subschemes $Y \rmono X$ which sends $I$ to $\spec \quot{A}{I}$.
\end{corollary}
\backmatter
\end{document}
