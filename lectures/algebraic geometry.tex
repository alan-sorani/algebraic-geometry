\documentclass[10pt,a4paper,twoside,openany,hidelinks]{book}

%JAPANESE
\usepackage{fontspec}
\defaultfontfeatures{Ligatures={NoCommon, NoDiscretionary, NoHistoric, NoRequired, NoContextual}}
\setmainfont{CMU Serif}
\usepackage{xeCJK}
\xeCJKsetup{CJKmath=true}
\setCJKmainfont{MS Mincho} % or some other Japanese font

\usepackage{maths}
\usepackage{stylish}

\usepackage{tkz-graph}
\GraphInit[vstyle = Shade]

\title{Lecture Notes to Fundamental Concepts in Algebraic Geometry \\ \large{Spring 2020, Hebrew University of Jerusalem}}
\author{Ariel Schnidman\\ \large{Typed by Elad Tzorani}}
\date{\today}

\usepackage{lipsum}
\usepackage{stmaryrd}

\tikzset{EdgeStyle/.style   = {thick,
                               double          = orange,
                               double distance = 1pt}}

\tikzset{
    labl0/.style={anchor=south, rotate=-30, inner sep=.5mm}
}

\tikzset{
    labl1/.style={anchor=south, rotate=30, inner sep=.5mm}
}

\tikzset{
    labl2/.style={anchor=south, rotate=60, inner sep=.5mm}
}

\tikzset{
    labl3/.style={anchor=south, rotate=90, inner sep=.5mm}
}

\tikzset{
    labl4/.style={anchor=south, rotate=-90, inner sep=.5mm}
}

\usepackage{scalerel,stackengine}
\stackMath
\newcommand\widecheck[1]{%
\savestack{\tmpbox}{\stretchto{%
  \scaleto{%
    \scalerel*[\widthof{\ensuremath{#1}}]{\kern-.6pt\bigwedge\kern-.6pt}%
    {\rule[-\textheight/2]{1ex}{\textheight}}%WIDTH-LIMITED BIG WEDGE
  }{\textheight}% 
}{0.5ex}}%
\stackon[1pt]{#1}{\scalebox{-1}{\tmpbox}}%
}
\parskip 1ex

\newcommand{\from}{\leftarrow}

\usepackage[nottoc]{tocbibind}

\begin{document}
\frontmatter
\frontpage{front}{0.8\textwidth}{}
\tableofcontents
\countlectures
\newpage

\chapter*{Preface}
\addcontentsline{toc}{chapter}{Preface} \markboth{Preface}{}

\section*{Technicalities}
\addcontentsline{toc}{section}{Technicalities} %\markboth{Technicalities}{}

These aren't formal notes related to the course and henceforward there is \emph{absolutely no guarantee} that the recorded material is in correspondence with the course expectations, or that these notes lack any mistakes.\\
In fact, there probably are mistakes in the notes! I would highly appreciate if any comments or corrections were sent to me via email at \href{mailto:tzorani.elad@gmail.com}{tzorani.elad@gmail.com}.\\
Elad Tzorani.

\mainmatter

\chapter*{Goals of Algebraic Geometry}

The main goals of algebraic geometry are the following.

\begin{itemize}
\item To classify algebraic varieties up to birational isomorphism.
Either by enumeration of the varieties or classification by properties.
\item
Be able to tell whether two algebraic varieties are (birationally) isomorphic. This is done through different tools:
\begin{itemize}
\item Cohomology (related to the Hodge Conjecture).
\item Vector bundles (related to K-theory).
\item Algebraic Cycles (subvarieties).
\item Coherent sheaves (derived category).
\item Topology.
\end{itemize}
\end{itemize}

\chapter{Sheaves}

\section{Basic Definitions}

Let $X \in \Top$.

\begin{definition}
Let $\Top_X$ be the category of open sets on $X$ with inclusions as morphisms.
\end{definition}

\begin{notation}
Denote by $\Ab$ the category of abelian groups.
\end{notation}

\begin{definition}[Presheaf]
A \stress{presheaf (of groups)} on $X$ is a contravariant functor
\[F \colon \Top_X \to \Ab \text{.}\]
\end{definition}

Concretely, to give a presheaf we must give the following:
\begin{enumerate}
\item For each $U \subseteq X$ open, an abelian group $F\prs{U}$.
\item For inclusion $V \rmono U$ of open sets, a group homomorphism $F_{U,V} \colon F\prs{U} \to F\prs{V}$.
\end{enumerate}
Under the conditions:
\begin{enumerate}
\item $F_{U,U} = \id_{F\prs{U}}$.
\item If $W \rmono V \rmono U$ then $W \rmono U$ so the induced maps
\[
\begin{tikzcd}
F\prs{U} \arrow[r] \arrow[rr, bend right = 30] & F\prs{V} \arrow[r] & F\prs{W}
\end{tikzcd}
\]
commute.
\end{enumerate}

\begin{examples}
\begin{enumerate}
\item Let \[F\prs{U} \ceq \hom\prs{U, \RR} = \set{f \colon U \to \RR} \text{.}\]
If $V \subseteq U$ are open,
\begin{align*}
F\prs{U} &\xrightarrow{F_{U,V}} F\prs{V} \\
f &\mapsto \rest{f}{V} \hphantom{F\prs{V}} \text{.}
\end{align*}

This is actually also a presheaf of rings on $X$.

\item \stress{The constant presheaf:} Let $A \in \Ab$ and define
\[F\prs{U} \ceq A, \quad F_{U,V} = \id_A \text{.}\]

\item Let $k$ be an algebraically closed field and $X$ an algebraic variety over $k$. We can think $X \subseteq \PP_k^n$.

Define $\Oo_X\prs{U} \ceq \set{f \colon U \to k}{\text{$f$ is regular}}$.
By ``regular'' we mean that for all $P \in U$ there's a neighbourhood $V \subseteq U$ of $P$ such that
\[\rest{f}{V} = \frac{g}{h}\]
with
$g,h \in k\brs{x_0, \ldots, x_n}$.

\item Let $X = \PP^1$.
Then
$\Oo_{\PP^1}\prs{\PP^1} = k$.
Also
$\Oo_{\PP^1} \prs{\PP^1 \setminus \set{0}} = \set{\frac{g\prs{x,y}}{h\prs{x,y}}}{\substack{g,h \in k\brs{x,y}\\\forall \prs{x,y} \neq \prs{0,1} \colon h\prs{x,y} \neq 0}} = \set{\frac{g\prs{x,y}}{x^{\deg\prs{g}}}}$.
\end{enumerate}
\end{examples}

\begin{definition}[Section, Restriction]
For any $F$ as above, $s \in F\prs{U}$ is called a \stress{section}.
$F_{U,V}\prs{s}$ is written as $\rest{s}{V}$ and called \stress{restriction of $s$ to $V$}.
\end{definition}

\begin{definition}[Global Section]
$F\prs{X} \ceq \Gamma\prs{X,F}$ is the \stress{global section of $F$}.
\end{definition}

\begin{remark}
Why ``section''? Think about $\Oo_X\prs{U}$ and view the diagram
\[
\begin{tikzcd}
U \times k \arrow[d,"\pi_U"] \arrow[r, hook] & X \times k \arrow[d,"\pi"] \\
U \arrow[u, "\tilde{s}", bend left = 40] \arrow[r, hook] & X
\end{tikzcd}
\]
and call $X \times k$ the \stress{trivial line bundle}.

Take $s \in \Oo_X\prs{U}$ which induces a map
\begin{align*}
\tilde{s} \colon U &\to U \times k \\
x &\mapsto \prs{X, s\prs{x}} \text{.}
\end{align*}
Then $\tilde{s}$ is a {section of $\pi$ over $U$}
in the sense that $\pi_U \circ \tilde{s} = \id_U$.
\end{remark}

The idea of sheaves is that global sections are difficult to study topologically, but $F\prs{U}$ is easier to study for $U$ small, and computation on smaller $U$ can imply properties of $F\prs{U}$.



\begin{comment}
\begin{thebibliography}{2}
\bibitem{context} 
Emily Riehl. 
\textit{Category Theory in Context}. 

\bibitem{nlab}
\textit{nLab - Online Wiki with Focus on Category Theory and Homotopy Theory}.\\
\href{https://ncatlab.org/nlab/show/HomePage}{https://ncatlab.org/nlab/show/HomePage}
\end{thebibliography}
\end{comment}

\backmatter
\end{document}